%Regularized Regression on AutoMPG
\documentclass{article}
\usepackage{amsmath} % For advanced math features like \text, \frac, etc.
\usepackage[utf8]{inputenc} % To handle various characters
\begin{document}
\begin{table}[h]
\centering
\caption{Lasso and Ridge Regression AutoMPG results}
\label{tab:Regularized Regression on AutoMPG}
\begin{tabular}{|c|c|c|} \hline
Ridge & Lasso \\ \hline \hline
rSq &0.809255 &  0.804643 \\ \hline
rSqBar &0.806283 &       0.802112 \\ \hline
sst &23819.0 &   23819.0 \\ \hline
sse &4543.35 &   4653.21 \\ \hline
sde &3.40878 &   3.44970 \\ \hline
mse0 &11.5902 &  11.8704 \\ \hline
rmse &3.40443 &  3.44535 \\ \hline
mae &2.61826 &   2.62953 \\ \hline
smape &64.8110 & 12.0246 \\ \hline
m &392.000 &     392.000 \\ \hline
dfr &6.00000 &   5.00000 \\ \hline
df &385.000 &    386.000 \\ \hline
fStat &272.234 & 317.974 \\ \hline
aic &-1022.45 &  -1029.14 \\ \hline
bic &-994.656 &  -1005.31 \\ \hline
\end{tabular}
\end{table}
\end{document}

% Regularized Regression on House Prices
\documentclass{article}
\usepackage{amsmath} % For advanced math features like \text, \frac, etc.
\usepackage[utf8]{inputenc} % To handle various characters
\begin{document}
\begin{table}[h]
\centering
\caption{Lasso and Ridge Regression House Prices results}
\label{tab:Regularized Regression on House Prices}
\begin{tabular}{|c|c|c|} \hline
Ridge & Lasso \\ \hline \hline
rSq &0.998517 &  0.981186 \\ \hline
rSqBar &0.998507 &       0.981072 \\ \hline
sst &6.42325e+13 &       6.42325e+13 \\ \hline
sse &9.52491e+10 &       1.20848e+12 \\ \hline
sde &9764.45 &   22644.0 \\ \hline
mse0 &9.52491e+07 &      1.20848e+09 \\ \hline
rmse &9759.56 &  34763.2 \\ \hline
mae &7740.46 &   28672.9 \\ \hline
smape &9.75718 & 5.78522 \\ \hline
m &1000.00 &     1000.00 \\ \hline
dfr &7.00000 &   6.00000 \\ \hline
df &992.000 &    993.000 \\ \hline
fStat &95425.2 & 8631.07 \\ \hline
aic &-10588.9 &  -12111.7 \\ \hline
bic &-10549.7 &  -12077.3 \\ \hline
\end{tabular}
\end{table}
\end{document}

% Regularized Regression on Insurance-
\documentclass{article}
\usepackage{amsmath} % For advanced math features like \text, \frac, etc.
\usepackage[utf8]{inputenc} % To handle various characters
\begin{document}
\begin{table}[h]
\centering
\caption{Lasso and Ridge Regression for Insurance Charges results}
\label{tab:Regularized Regression on Insurance Charges}
\begin{tabular}{|c|c|c|} \hline
Ridge & Lasso \\ \hline \hline
rSq &0.750913 &  0.723537 \\ \hline
rSqBar &0.749414 &       0.722082 \\ \hline
sst &1.96074e+11 &       1.96074e+11 \\ \hline
sse &4.88396e+10 &       5.42073e+10 \\ \hline
sde &6043.94 &   6358.54 \\ \hline
mse0 &3.65019e+07 &      4.05137e+07 \\ \hline
rmse &6041.68 &  6365.04 \\ \hline
mae &4171.66 &   4367.97 \\ \hline
smape &68.9369 & 37.6686 \\ \hline
m &1338.00 &     1338.00 \\ \hline
dfr &8.00000 &   7.00000 \\ \hline
df &1329.00 &    1330.00 \\ \hline
fStat &500.810 & 497.252 \\ \hline
aic &-13529.8 &  -13601.5 \\ \hline
bic &-13483.0 &  -13559.9 \\ \hline
\end{tabular}
\end{table}
\end{document}

