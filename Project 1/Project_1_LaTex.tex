\documentclass{article}
\usepackage{graphicx} % Required for inserting images
\usepackage[utf8]{inputenc}
\usepackage{fancyhdr}
\usepackage[shortlabels]{enumitem}
\usepackage{amsmath, amsfonts, amsthm, amssymb, mathrsfs}
\usepackage{tikz}
\usepackage{subcaption}
\usepackage[left=1cm, right=1cm, top=2cm, bottom=2cm]{geometry}
\usepackage{stackengine}
\usepackage{tikz-cd}
\usetikzlibrary{cd}
\usetikzlibrary{calc}
\usepackage{lipsum}
\usepackage{adjustbox}
%\usepackage{cleveref}
\numberwithin{equation}{subsection}

\usepackage{scalerel,stackengine}
\stackMath
\newcommand\reallywidehat[1]{%
\savestack{\tmpbox}{\stretchto{%
  \scaleto{%
    \scalerel*[\widthof{\ensuremath{#1}}]{\kern-.6pt\bigwedge\kern-.6pt}%
    {\rule[-\textheight/2]{1ex}{\textheight}}%WIDTH-LIMITED BIG WEDGE
  }{\textheight}% 
}{0.5ex}}%
\stackon[1pt]{#1}{\tmpbox}%
}

\newcommand{\crly}[1]{\left\{#1\right\}}
\newcommand{\sq}[1]{\left[#1\right]}
\newcommand{\cur}[1]{\left(#1\right)}
\newcommand{\lrangle}[1]{\left\langle#1\right\rangle}
\newcommand{\abs}[1]{\left|#1\right|}
\newcommand{\nrm}[1]{\left|\left|#1\right|\right|}
\newcommand{\ceil}[1]{\left\lceil#1\right\rceil}
\newcommand{\floor}[1]{\left\lfloor#1\right\rfloor}

\newcommand{\suml}[2]{\sum\limits_{#1}^{#2}}
\newcommand{\prodl}[2]{\prod\limits_{#1}^{#2}}
\newcommand{\bigcupl}[2]{\bigcup\limits_{#1}^{#2}}
\newcommand{\bigcapl}[2]{\bigcap\limits_{#1}^{#2}}
\newcommand{\liml}[1]{\lim\limits_{#1}}
\newcommand{\supl}[1]{\sup\limits_{#1}}
\newcommand{\infl}[1]{\inf\limits_{#1}}

\newcommand{\N}{\mathbb{N}}
\newcommand{\R}{\mathbb{R}}
\newcommand{\Z}{\mathbb{Z}}
\newcommand{\C}{\mathbb{C}}
\newcommand{\Q}{\mathbb{Q}}
\newcommand{\mf}[1]{\mathfrak{#1}}
\newcommand{\mc}[1]{\mathcal{#1}}
\newcommand{\del}{\partial}
%\newcommand{\ker}{\text{ker}}
\newcommand{\Ker}{\textup{Ker}}
\newcommand{\Coker}{\textup{Coker}}
\newcommand{\im}{\textup{im}}
\newcommand{\IM}{\textup{Im}}
\newcommand{\id}{\textup{id}}
\newcommand{\Hom}[2]{\textup{Hom}_{#1}\cur{#2}}
\newcommand{\ind}[2]{\textup{ind}_{#1}^{#2}}
\newcommand{\res}[2]{\textup{res}_{#1}^{#2}}
\newcommand{\Rind}[3]{\textup{R}^{#1}\textup{ind}_{#2}^{#3}}
\newcommand{\rRind}[4]{\textup{R}_{#1}^{#2}\textup{ind}_{#3}^{#4}}
\newcommand{\soc}{\textup{soc}}
\newcommand{\Ext}[2]{\textup{Ext}_{#1}^{#2}}
\newcommand{\Tor}[2]{\textup{Tor}_{#1}^{#2}}
\newcommand{\Tot}{\textup{Tot}}
\newcommand{\Mod}[1]{\textup{Mod}\cur{#1}}
%\newcommand{\mod}[1]{\textup{mod}\cur{#1}}
\newcommand{\St}{\textup{St}}
\newcommand{\Char}[1]{\textup{char}\cur{#1}}

%\newcommand{\a}{\alpha}
%\newcommand{\b}{\beta}
%\newcommand{\d}{\delta}
\newcommand{\e}{\varepsilon}
%\newcommand{\D}{\Delta}
%\newcommand{\E}{\varEpsilon}

\makeatletter
\newcommand*{\medcdot}{\mathpalette\medcdot@{0.5}}
\newcommand*\medcdot@[2]{\mathbin{\vcenter{\hbox{\scalebox{#2}{$\m@th#1\bullet$}}}}}
\makeatother

\makeatletter
\newcommand*{\bigcdot}{\mathpalette\bigcdot@{0.75}}
\newcommand*\bigcdot@[2]{\mathbin{\vcenter{\hbox{\scalebox{#2}{$\m@th#1\bullet$}}}}}
\makeatother

%\renewcommand{\sum}[2]{\sum\limits_{#1}^{#2}}

\renewcommand{\qedsymbol}{$\blacksquare$}

\newtheorem{theorem}{Theorem}[section]
\newtheorem{lemma}[theorem]{Lemma}
\newtheorem{proposition}[theorem]{Proposition}
\newtheorem{corollary}[theorem]{Corollary}
\newtheorem{conjecture}[theorem]{Conjecture}
\newtheorem{observation}[theorem]{Observation}
\newtheorem{claim}[theorem]{Claim}
\newtheorem{question}[theorem]{Question}
\newtheorem{exercise}[theorem]{Exercise}
\newtheorem{example}[theorem]{Example}

\newcommand{\sref}[1]{\textup{sequence}~(\ref{#1})}
\newcommand{\Sref}[1]{\textup{Sequence}~(\ref{#1})}

\newcommand{\crefdefpart}[2]{%
  \hyperref[#2]{\namecref{#1}~\labelcref*{#1}~\ref*{#2}}%
}

\newcommand{\Crefdefpart}[2]{%
  \nameCref{#1}~\hyperref[#2]{\labelcref*{#1}~\ref*{#2}}%
}

\usepackage[backend=bibtex, style=alphabetic, sorting=nyt, maxnames=99]{biblatex}

\usepackage[colorlinks=true, urlcolor=blue, linkcolor=blue, citecolor=blue]{hyperref}

\usepackage{cleveref}

\usepackage{booktabs}

\usepackage{float}



%%%%removes the autamitic 'p' when citing a page
\DeclareFieldFormat{postnote}{#1}
\DeclareFieldFormat{multipostnote}{#1}

\title{Project 1}
\author{Manager: Brendan McDonnel \phantom{thisisatest}\\
Masum Billah\\
Madhu Chencharapu\\
Gabriel Loos\\
Roshan Ravichandran}
\date{February 2026}

\begin{document}

\maketitle

\section{Introduction}

In this project, we will explore three different data sets using multiple linear regression, ridge regression, lasso, and transformed regression (specifically sqrt, log1p, and Box-Cox). The three data sets we will be using are 

\begin{itemize}
	\item \textbf{Auto MPG:} This data set contains information about different cars. The goal is predict the miles per gallon (MPG) given the other characteristics of the car. The data set can be found at \url{https://archive.ics.uci.edu/dataset}
	\item \textbf{House Prices:} This data set contains information about different houses. The goal is to predict the price of the house given the other characteristics. The data set can be found at \url{https://www.kaggle.com/datasets/prokshitha/home-value-insights}
	\item \textbf{Insurance Charges:} This data set contains information about different people. The goal is to predict a persons insurance charges given their other characteristics. The data set can be found at \url{https://www.kaggle.com/datasets/mirichoi0218/insurance}
\end{itemize}

\section{Linear Regression} \label{Linear Regression}

In this section, multiple linear regression is applied to the Auto MPG, House Prices, and Insurance Charges data sets. We evaluated the model using in-sample, out-of-sample, and cross validation. Finally, we applied forward selection, backward elimination, and stepwise selection to identify which features are optimal for explanation of the response variable.

\subsection{Auto MPG}

\Cref{tab:Scalation - AutoMPG Linear Regression} presents the quality of fit metrics for the in-sample and out-of-sample evaluations using scalation, while \Cref{tab:Statsmodels - Auto MPG Linear Regression} presents the same metrics using statsmodels. We can see that regression is performing decently, and that the main statistics are similar for both scalation and mathstats.

\begin{table}[H]
\centering
\caption{Scalation - AutoMPG Linear Regression}
\label{tab:Scalation - AutoMPG Linear Regression}
\begin{tabular}{|c|c|c|} \hline 
Metric & In-Sample & 80-20 Split \\ \hline \hline 
rSq &0.809255 & 0.822842 \\ \hline 
rSqBar &0.806283 &      0.820081 \\ \hline 
sst &23819.0 &  4731.23 \\ \hline 
sse &4543.35 &  838.174 \\ \hline 
sde &3.40878 &  3.29026 \\ \hline 
mse0 &11.5902 & 10.7458 \\ \hline 
rmse &3.40443 & 3.27808 \\ \hline 
mae &2.61826 &  2.48735 \\ \hline 
smape &12.0589 &  11.8858 \\ \hline 
m &392.000 &    78.0000 \\ \hline 
dfr &6.00000 &  6.00000 \\ \hline 
df &385.000 &   385.000 \\ \hline 
fStat &272.234 &        298.034 \\ \hline 
aic &-1022.45 & -189.284 \\ \hline 
bic &-994.656 & -172.787 \\ \hline 
\end{tabular}
\end{table}

\begin{table}[H]
\centering
\caption{Statsmodels - Auto MPG Linear Regression}
\label{tab:Statsmodels - Auto MPG Linear Regression}
\begin{tabular}{|c|c|c|}\hline
Metric & In-Sample & 80-20 Split \\ \hline \hline
rSq & 0.8093 & 0.7942 \\ \hline
rSqBar & 0.8063 & 0.7801 \\ \hline
sst & 23818.9935 & 4032.2061 \\ \hline
sse & 4543.3470 & 829.6873 \\ \hline
sde & 3.4352 & 3.3713 \\ \hline
mse0 & 11.8009 & 10.5024 \\ \hline
rmse & 3.4352 & 3.2407 \\ \hline
mae & 2.6183 & 2.5039 \\ \hline
smape & 12.0589 & 12.3880 \\ \hline
m & 392.0000 & 79.0000 \\ \hline
dfr & 6.0000 & 6.0000 \\ \hline
df & 385.0000 & 73.0000 \\ \hline
fStat & 272.2341 & 46.9622 \\ \hline
aic & 2086.9095 & 197.7765 \\ \hline
bic & 2114.7083 & 211.9932 \\ \hline
\end{tabular}
\end{table}

\Cref{fig:Scalation - AutoMPG reg} shows that plots for the predicted $y$-values vs. the actual $y$-values from scalation, while \Cref{fig:Statsmodels - AutoMPG reg} show the same from mathstats. Again the results are very similar.

\begin{figure}[H]
    \centering
    \captionsetup{justification=centering}
    \begin{subfigure}[b]{0.45\textwidth}
        \centering
        \includegraphics[width=\textwidth]{Auto_MPG/scalation_reg_In_Sample.png} 
        %\caption{Description of left image}
        %\label{fig:left}
    \end{subfigure}
    \hfill % Adds flexible space between the images
    \begin{subfigure}[b]{0.45\textwidth}
        \centering
        \includegraphics[width=\textwidth]{Auto_MPG/scalation_reg_80_20.png}
        %\caption{Description of right image}
        %\label{fig:right}
    \end{subfigure}
    \caption{Scalation - Auto MPG Regression\\ Left: In Sample Predictions\\ Right: 80-20 Out of Sample Predictions\\ yy black/actual vs. yp red/predicted}
    \label{fig:Scalation - AutoMPG reg}
\end{figure}

\begin{figure}[H]
    \centering
    \captionsetup{justification=centering}
    \begin{subfigure}[b]{0.45\textwidth}
        \centering
        \includegraphics[width=1.1\textwidth]{Auto_MPG/statsmodels_reg_In_Sample.png} 
        %\caption{Description of left image}
        %\label{fig:left}
    \end{subfigure}
    \hfill % Adds flexible space between the images
    \begin{subfigure}[b]{0.45\textwidth}
        \centering
        \includegraphics[width=1.1\textwidth]{Auto_MPG/statsmodels_reg_80_20.png}
        %\caption{Description of right image}
        %\label{fig:right}
    \end{subfigure}
    \caption{Statsmodels - Auto MPG Regression\\ Left: In Sample Predictions\\ Right: 80-20 Out of Sample Predictions\\ yy black/actual vs. yp red/predicted}
    \label{fig:Statsmodels - AutoMPG reg}
\end{figure}

Next, we performed 5 fold cross validation. \Cref{tab: Scalation - Auto MPG Linear Regression CV,tab:Statsmodels - Auto MPG Linear Regression CV} show the resulting quality of fit metrics from scalation and statsmodels respectively. Again we see that the results are similar.

\begin{table}[H]
\centering
\caption{Scalation - Auto MPG Linear Regression CV}
\label{tab: Scalation - Auto MPG Linear Regression CV}
\begin{tabular}{|c|c|c|c|c|c|c|} \hline
Name & num & min & max & mean & stdev & interval \\ \hline \hline
rSq & 5 & 0.788 & 0.823 & 0.798 & 0.014 & 0.018 \\ \hline
rSqBar & 5 & 0.785 & 0.820 & 0.795 & 0.014 & 0.018 \\ \hline
sst & 5 & 3962.818 & 5671.580 & 4700.481 & 620.767 & 770.935 \\ \hline
sse & 5 & 824.554 & 1176.435 & 950.494 & 142.696 & 177.215 \\ \hline
sde & 5 & 3.177 & 3.738 & 3.431 & 0.226 & 0.281 \\ \hline
mse0 & 5 & 10.571 & 15.083 & 12.186 & 1.829 & 2.272 \\ \hline
rmse & 5 & 3.251 & 3.884 & 3.483 & 0.256 & 0.318 \\ \hline
mae & 5 & 2.487 & 2.850 & 2.689 & 0.151 & 0.188 \\ \hline
smape & 5 & 11.886 & 12.905 & 12.372 & 0.427 & 0.530 \\ \hline
m & 5 & 78.000 & 78.000 & 78.000 & 0.000 & 0.000 \\ \hline
dfr & 5 & 6.000 & 6.000 & 6.000 & 0.000 & 0.000 \\ \hline
df & 5 & 385.000 & 385.000 & 385.000 & 0.000 & 0.000 \\ \hline
fStat & 5 & 239.054 & 298.034 & 254.430 & 24.517 & 30.448 \\ \hline
aic & 5 & -205.110 & -188.647 & -194.539 & 6.676 & 8.291 \\ \hline
bic & 5 & -188.613 & -172.150 & -178.042 & 6.676 & 8.291 \\ \hline
\end{tabular}
\end{table}

\begin{table}[H]
\centering
\caption{Statsmodels - Auto MPG Linear Regression CV}
\label{tab:Statsmodels - Auto MPG Linear Regression CV}
\begin{tabular}{|c|c|c|c|c|c|}\hline
Name & In-num folds & min & max & mean & stdev \\ \hline \hline
rSq & 5 & 0.7654 & 0.8282 & 0.8010 & 0.0216 \\ \hline
rSqBar & 5 & 0.7491 & 0.8163 & 0.7873 & 0.0231 \\ \hline
sst & 5 & 4032.2061 & 5792.1365 & 4724.2751 & 617.8288 \\ \hline
sse & 5 & 745.1709 & 1359.0577 & 947.8346 & 213.7052 \\ \hline
sde & 5 & 3.2171 & 4.3446 & 3.5980 & 0.3912 \\ \hline
mse0 & 5 & 9.5535 & 17.4238 & 12.0958 & 2.7627 \\ \hline
rmse & 5 & 3.0909 & 4.1742 & 3.4576 & 0.3756 \\ \hline
mae & 5 & 2.5039 & 3.1904 & 2.6786 & 0.2601 \\ \hline
smape & 5 & 11.2379 & 14.1325 & 12.3805 & 0.9795 \\ \hline
m & 5 & 78.0000 & 79.0000 & 78.4000 & 0.4899 \\ \hline
dfr & 5 & 6.0000 & 6.0000 & 6.0000 & 0.0000 \\ \hline
df & 5 & 72.0000 & 73.0000 & 72.4000 & 0.4899 \\ \hline
fStat & 5 & 39.1425 & 57.8612 & 49.2658 & 6.3699 \\ \hline
aic & 5 & 188.0386 & 234.9114 & 205.6271 & 15.7223 \\ \hline
bic & 5 & 202.1788 & 249.0516 & 219.7980 & 15.7128 \\ \hline
\end{tabular}
\end{table}

Finally, we apply forward selection, backward elimination, and stepwise selection to determine which variables best explain response variable. \Cref{fig:Scalation - AutoMPG reg FS} shows plots for $R^2$, $\overline{R^2}$, sMAPE, $R^2$ cv, and AIC vs. $n$ (the number of variables selected) when utilizing forward selection. From the left plot, it is clear that only $2$ variables are needed to explain the response variable (even just $1$ variable is not too bad either). Those variables $2$ variables are weight and modelyear. The order in which the variables were chosen was 1. weight, 2. modelyear, 3. cylinders, 4. acceleration, 5. horsepower, and 6. displacment.

One interesting thing to note is how the graph of AIC vs. $n$ is always increasing.

\begin{figure}[H]
    \centering
    \captionsetup{justification=centering}
    \begin{subfigure}[b]{0.45\textwidth}
        \centering
        \includegraphics[width=\textwidth]{Auto_MPG/scalation_reg_FS_R2.png} 
        %\caption{Description of left image}
        %\label{fig:left}
    \end{subfigure}
    \hfill % Adds flexible space between the images
    \begin{subfigure}[b]{0.45\textwidth}
        \centering
        \includegraphics[width=\textwidth]{Auto_MPG/scalation_reg_FS_aic.png}
        %\caption{Description of right image}
        %\label{fig:right}
    \end{subfigure}
    \caption{Scalation - Auto MPG Regression Forward Selection\\ Left: $R^2$ vs. $n$\\ Right: aic vs. $n$}
    \label{fig:Scalation - AutoMPG reg FS}
\end{figure}

\Cref{fig:Scalation - AutoMPG reg BE} shows plots for $R^2$, $\overline{R^2}$, sMAPE, $R^2$ cv, and AIC vs. $n$ (the number of variables selected) when utilizing backward elimination. From the left plot, it is again clear that only $2$ variables are needed to explain the response variable (and again even just $1$ variable is not too bad either). Here, both backward and forward selection agree with each other in their selection of variables, so if you were to remove one variable at a time, in order you would remove 1. displacment, 2. horsepower, 3. acceleration, 4. cylinders, 5. modelyear, and 6. weight. This is the reverse order of forward selection, meaning that they agree.

\begin{figure}[H]
    \centering
    \captionsetup{justification=centering}
    \begin{subfigure}[b]{0.45\textwidth}
        \centering
        \includegraphics[width=\textwidth]{Auto_MPG/scalation_reg_BE_R2.png} 
        %\caption{Description of left image}
        %\label{fig:left}
    \end{subfigure}
    \hfill % Adds flexible space between the images
    \begin{subfigure}[b]{0.45\textwidth}
        \centering
        \includegraphics[width=\textwidth]{Auto_MPG/scalation_reg_BE_aic.png}
        %\caption{Description of right image}
        %\label{fig:right}
    \end{subfigure}
    \caption{Scalation - Auto MPG Regression Backward Elimination\\ Left: $R^2$ vs. $n$\\ Right: aic vs. $n$}
    \label{fig:Scalation - AutoMPG reg BE}
\end{figure}

Last, \Cref{fig:Scalation - AutoMPG reg SS} shows plots for $R^2$, $\overline{R^2}$, sMAPE, $R^2$ cv, and AIC vs. $n$ (the number of variables selected) when utilizing stepwise selection. Here, since stepwise can move forward and backwards which results in multiple different metric evaluations for models that use the same number of variables, (I believe) the evaluation that is plotted for $n$ variables is the best evaluation that stepwise selection found for all the models with $n$ variables that it tested. Despite this, we still find that if we want to get the best models recommended from stepwise, we should add in the following order, 1. weight, 2. modelyear, 3. cylinders, 4. acceleration, and 5. horsepower. Note that this almost agrees with forward selection and backward elimination, but we do not add displacement as stepwise decided that moving was not worthwhile.

\begin{figure}[H]
    \centering
    \captionsetup{justification=centering}
    \begin{subfigure}[b]{0.45\textwidth}
        \centering
        \includegraphics[width=\textwidth]{Auto_MPG/scalation_reg_SS_R2.png} 
        %\caption{Description of left image}
        %\label{fig:left}
    \end{subfigure}
    \hfill % Adds flexible space between the images
    \begin{subfigure}[b]{0.45\textwidth}
        \centering
        \includegraphics[width=\textwidth]{Auto_MPG/scalation_reg_SS_aic.png}
        %\caption{Description of right image}
        %\label{fig:right}
    \end{subfigure}
    \caption{Scalation - Auto MPG Regression Stepwise Selection\\ Left: $R^2$ vs. $n$\\ Right: aic vs. $n$}
    \label{fig:Scalation - AutoMPG reg SS}
\end{figure}

\subsection{House Prices}

\Cref{tab:Scalation - House Price Linear Regression} presents the quality of fit metrics for the in-sample and out-of-sample evaluations using scalation, while \Cref{tab:Statsmodels - House Price Linear Regression} presents the same metrics using statsmodels. We can see that regression is performing extremely well, almost perfectly capturing the data. The main statistics are similar for both scalation and mathstats.

\begin{table}[H]
\centering
\caption{Scalation - House Price Linear Regression}
\label{tab:Scalation - House Price Linear Regression}
\begin{tabular}{|c|c|c|} \hline 
Metric & In-Sample & 80-20 Split \\ \hline \hline 
rSq &0.998516 & 0.998649 \\ \hline 
rSqBar &0.998507 &      0.998641 \\ \hline 
sst &6.42325e+13 &      1.33700e+13 \\ \hline 
sse &9.53030e+10 &      1.80638e+10 \\ \hline 
sde &9767.21 &  9501.56 \\ \hline 
mse0 &9.53030e+07 &     9.03190e+07 \\ \hline 
rmse &9762.32 & 9503.63 \\ \hline 
mae &7747.66 &  7484.48 \\ \hline 
smape &1.57791 &        1.60847 \\ \hline 
m &1000.00 &    200.000 \\ \hline 
dfr &6.00000 &  6.00000 \\ \hline 
df &993.000 &   993.000 \\ \hline 
fStat &111378 & 122330 \\ \hline 
aic &-10591.2 & -2101.67 \\ \hline 
bic &-10556.9 & -2078.59 \\ \hline 
\end{tabular}
\end{table}


\begin{table}[H]
\centering
\caption{Statsmodels - House Price Linear Regression}
\label{tab:Statsmodels - House Price Linear Regression}
\begin{tabular}{|c|c|c|}\hline
Metric & In-Sample & 80-20 Split \\ \hline \hline
rSq & 0.9985 & 0.9984 \\ \hline
rSqBar & 0.9985 & 0.9984 \\ \hline
sst & 64232463468052.5469 & 12891771417242.6445 \\ \hline
sse & 95249090298.3967 & 20286959701.1265 \\ \hline
sde & 9798.8381 & 10252.5012 \\ \hline
mse0 & 96017228.1234 & 101434798.5056 \\ \hline
rmse & 9798.8381 & 10071.4844 \\ \hline
mae & 7740.4301 & 8174.5836 \\ \hline
smape & 1.5774 & 1.6620 \\ \hline
m & 1000.0000 & 200.0000 \\ \hline
dfr & 7.0000 & 7.0000 \\ \hline
df & 992.0000 & 193.0000 \\ \hline
fStat & 95425.1583 & 17493.2676 \\ \hline
aic & 21225.8831 & 3700.9854 \\ \hline
bic & 21265.1451 & 3724.0736 \\ \hline
\end{tabular}
\end{table}

\Cref{fig:Scalation - Housing reg} shows that plots for the predicted $y$-values vs. the actual $y$-values from scalation, while \Cref{fig:Statsmodels - Housing reg} show the same from mathstats. Again the results are very similar.

\begin{figure}[H]
    \centering
    \captionsetup{justification=centering}
    \begin{subfigure}[b]{0.45\textwidth}
        \centering
        \includegraphics[width=\textwidth]{Housing/scalation_reg_In_Sample.png} 
        %\caption{Description of left image}
        %\label{fig:left}
    \end{subfigure}
    \hfill % Adds flexible space between the images
    \begin{subfigure}[b]{0.45\textwidth}
        \centering
        \includegraphics[width=\textwidth]{Housing/scalation_reg_80_20.png}
        %\caption{Description of right image}
        %\label{fig:right}
    \end{subfigure}
    \caption{Scalation - House Price Regression\\ Left: In Sample Predictions\\ Right: 80-20 Out of Sample Predictions\\ yy black/actual vs. yp red/predicted}
    \label{fig:Scalation - Housing reg}
\end{figure}

\begin{figure}[H]
    \centering
    \captionsetup{justification=centering}
    \begin{subfigure}[b]{0.45\textwidth}
        \centering
        \includegraphics[width=1.1\textwidth]{Housing/statsmodels_reg_In_Sample.png} 
        %\caption{Description of left image}
        %\label{fig:left}
    \end{subfigure}
    \hfill % Adds flexible space between the images
    \begin{subfigure}[b]{0.45\textwidth}
        \centering
        \includegraphics[width=1.1\textwidth]{Housing/statsmodels_reg_80_20.png}
        %\caption{Description of right image}
        %\label{fig:right}
    \end{subfigure}
    \caption{Statsmodels - House Price Regression\\ Left: In Sample Predictions\\ Right: 80-20 Out of Sample Predictions\\ yy black/actual vs. yp red/predicted}
    \label{fig:Statsmodels - Housing reg}
\end{figure}

Next, we performed 5 fold cross validation. \Cref{tab: Scalation - House Price Linear Regression CV,tab:Statsmodels - House Price Linear Regression CV} show the resulting quality of fit metrics from scalation and statsmodels respectively. Again we see that the results are similar.

\begin{table}[H]
\centering
\caption{Scalation - House Price Linear Regression CV}
\label{tab: Scalation - House Price Linear Regression CV}
\begin{tabular}{|c|c|c|c|c|c|c|} \hline 
Name &  num folds & min & max & mean & stdev & interval \\ \hline \hline 
rSq & 5 & 0.998 & 0.999 & 0.998 & 0.000 & 0.000 \\ \hline 
rSqBar & 5 & 0.998 & 0.999 & 0.998 & 0.000 & 0.000 \\ \hline 
sst & 5 & 11805100457351.200 &13369989427686.710 & 12829460198984.865 & 606159535374.424 & 752793908917.133 \\ \hline 
sse & 5 & 17790794374.705 & 22663619750.593 & 19394433560.564 & 2117948592.845 & 2630295668.134 \\ \hline 
sde & 5 &   9365.942 &  10608.530 &   9818.348 &    533.068 &    662.021 \\ \hline 
mse0 & 5 & 88953971.874 & 113318098.753 & 96972167.803 & 10589742.964 & 13151478.341 \\ \hline 
rmse & 5 &   9431.541 &  10645.097 &   9836.093 &    528.486 &    656.330 \\ \hline 
mae & 5 &   7418.536 &   8609.308 &   7817.189 &    534.033 &    663.219 \\ \hline 
smape & 5 & 1.408 &      1.804 &      1.592 &      0.141 &      0.176 \\ \hline 
m & 5 & 200.000 &    200.000 &    200.000 &      0.000 &      0.000 \\ \hline 
dfr & 5 & 6.000 &      6.000 &      6.000 &      0.000 &      0.000 \\ \hline 
df & 5 & 993.000 &    993.000 &    993.000 &      0.000 &      0.000 \\ \hline 
fStat & 5 & 96001.466 & 122329.999 & 110142.703 &  10843.190 &  13466.236 \\ \hline 
aic & 5 & -2127.278 &  -2100.155 &  -2109.082 &     11.789 &     14.641 \\ \hline 
bic & 5 & -2104.190 &  -2077.067 &  -2085.993 &     11.789 &     14.641 \\ \hline 
\end{tabular}
\end{table}

\begin{table}[H]
\centering
\caption{Statsmodels - House Price Linear Regression CV}
\label{tab:Statsmodels - House Price Linear Regression CV}
\begin{tabular}{|c|c|c|c|c|c|}\hline
Name & In-num folds & min & max & mean & stdev \\ \hline \hline
rSq & 5 & 0.9984 & 0.9986 & 0.9985 & 0.0001 \\ \hline
rSqBar & 5 & 0.9984 & 0.9986 & 0.9985 & 0.0001 \\ \hline
sst & 5 & 12384264865084.2500 & 13564576490393.4668 & 12842056251568.3359 & 395911885307.9127 \\ \hline
sse & 5 & 17238729843.0851 & 20996608313.5931 & 19277864710.9665 & 1296174615.0581 \\ \hline
sde & 5 & 9450.9176 & 10430.2788 & 9988.5569 & 337.6978 \\ \hline
mse0 & 5 & 86193649.2154 & 104983041.5680 & 96389323.5548 & 6480873.0753 \\ \hline
rmse & 5 & 9284.0535 & 10246.1232 & 9812.2003 & 331.7354 \\ \hline
mae & 5 & 7516.9137 & 8174.5836 & 7794.6342 & 224.6310 \\ \hline
smape & 5 & 1.5104 & 1.6620 & 1.5879 & 0.0540 \\ \hline
m & 5 & 200.0000 & 200.0000 & 200.0000 & 0.0000 \\ \hline
dfr & 5 & 7.0000 & 7.0000 & 7.0000 & 0.0000 \\ \hline
df & 5 & 193.0000 & 193.0000 & 193.0000 & 0.0000 \\ \hline
fStat & 5 & 17493.2676 & 20276.9346 & 18399.0661 & 1001.2402 \\ \hline
aic & 5 & 3668.4214 & 3707.8619 & 3690.3225 & 13.5977 \\ \hline
bic & 5 & 3691.5096 & 3730.9501 & 3713.4107 & 13.5977 \\ \hline
\end{tabular}
\end{table}

Finally, we apply forward selection, backward elimination, and stepwise selection to determine which variables best explain response variable. \Cref{fig:Scalation - Housing reg FS} shows plots for $R^2$, $\overline{R^2}$, sMAPE, $R^2$ cv, and AIC vs. $n$ (the number of variables selected) when utilizing forward selection. From the left plot, it is clear that only $1$ variable is needed to explain the response variable. The order in which the variables were chosen was 1. Square Footage, 2. Year Built, 3. Lot Size, 4. Num Bedrooms, 5. Num Bathrooms, and 6. Garage Size.

\begin{figure}[H]
    \centering
    \captionsetup{justification=centering}
    \begin{subfigure}[b]{0.45\textwidth}
        \centering
        \includegraphics[width=\textwidth]{Housing/scalation_reg_FS_R2.png} 
        %\caption{Description of left image}
        %\label{fig:left}
    \end{subfigure}
    \hfill % Adds flexible space between the images
    \begin{subfigure}[b]{0.45\textwidth}
        \centering
        \includegraphics[width=\textwidth]{Housing/scalation_reg_FS_aic.png}
        %\caption{Description of right image}
        %\label{fig:right}
    \end{subfigure}
    \caption{Scalation - House Prices Regression Forward Selection\\ Left: $R^2$ vs. $n$\\ Right: aic vs. $n$}
    \label{fig:Scalation - Housing reg FS}
\end{figure}

\Cref{fig:Scalation - Housing reg BE} shows plots for $R^2$, $\overline{R^2}$, sMAPE, $R^2$ cv, and AIC vs. $n$ (the number of variables selected) when utilizing backward elimination. From the left plot, it is again clear that only $1$ variable is needed to explain the response variable. Here, both backward and forward selection agree with each other in their selection of variables, so if you were to remove one variable at a time, in order you would remove 1. Garage Size, 2. Num Bathrooms, 3. Num Bedrooms, 4. Lot Size, 5. Year Built, and 6. Square Footage. This is the reverse order of forward selection, meaning that they agree.

\begin{figure}[H]
    \centering
    \captionsetup{justification=centering}
    \begin{subfigure}[b]{0.45\textwidth}
        \centering
        \includegraphics[width=\textwidth]{Housing/scalation_reg_BE_R2.png} 
        %\caption{Description of left image}
        %\label{fig:left}
    \end{subfigure}
    \hfill % Adds flexible space between the images
    \begin{subfigure}[b]{0.45\textwidth}
        \centering
        \includegraphics[width=\textwidth]{Housing/scalation_reg_BE_aic.png}
        %\caption{Description of right image}
        %\label{fig:right}
    \end{subfigure}
    \caption{Scalation - House Prices Regression Backward Elimination\\ Left: $R^2$ vs. $n$\\ Right: aic vs. $n$}
    \label{fig:Scalation - Housing reg BE}
\end{figure}

Last, \Cref{fig:Scalation - Housing reg SS} shows plots for $R^2$, $\overline{R^2}$, sMAPE, $R^2$ cv, and AIC vs. $n$ (the number of variables selected) when utilizing stepwise selection. We still find that if we want to get the best models recommended from stepwise, we should add in the following order, 1. Square Footage, 2. Year Built, 3. Lot Size, 4. Num Bedrooms, and 5. Num Bathrooms. Note that this almost agrees with forward selection and backward elimination, but we do not add Garage Size as stepwise decided that moving was not worthwhile.

\begin{figure}[H]
    \centering
    \captionsetup{justification=centering}
    \begin{subfigure}[b]{0.45\textwidth}
        \centering
        \includegraphics[width=\textwidth]{Housing/scalation_reg_SS_R2.png} 
        %\caption{Description of left image}
        %\label{fig:left}
    \end{subfigure}
    \hfill % Adds flexible space between the images
    \begin{subfigure}[b]{0.45\textwidth}
        \centering
        \includegraphics[width=\textwidth]{Housing/scalation_reg_SS_aic.png}
        %\caption{Description of right image}
        %\label{fig:right}
    \end{subfigure}
    \caption{Scalation - House Prices Regression Stepwise Selection\\ Left: $R^2$ vs. $n$\\ Right: aic vs. $n$}
    \label{fig:Scalation - Housing reg SS}
\end{figure}

\subsection{Insurance Charges}

\Cref{tab:Scalation - Insurance Charges Linear Regression} presents the quality of fit metrics for the in-sample and out-of-sample evaluations using scalation, while \Cref{tab:Statsmodels - Insurance Charges Linear Regression} presents the same metrics using statsmodels. We can see that regression is performing okay, and that the main statistics are similar for both scalation and mathstats.

\begin{table}[H]
\centering
\caption{Scalation - Insurance Charges Linear Regression}
\label{tab:Scalation - Insurance Charges Linear Regression}
\begin{tabular}{|c|c|c|} \hline 
Metric & In-Sample & 80-20 Split \\ \hline \hline 
rSq &0.750157 & 0.720005 \\ \hline 
rSqBar &0.748842 &      0.718531 \\ \hline 
sst &1.96074e+11 &      4.06432e+10 \\ \hline 
sse &4.89878e+10 &      1.13799e+10 \\ \hline 
sde &6053.11 &  6540.46 \\ \hline 
mse0 &3.66127e+07 &     4.26213e+07 \\ \hline 
rmse &6050.84 & 6528.50 \\ \hline 
mae &4179.54 &  4430.66 \\ \hline 
smape &37.9722 &        40.0602 \\ \hline 
m &1338.00 &    267.000 \\ \hline 
dfr &7.00000 &  7.00000 \\ \hline 
df &1330.00 &   1330.00 \\ \hline 
fStat &570.477 &        488.584 \\ \hline 
aic &-13533.8 & -2708.17 \\ \hline 
bic &-13492.2 & -2679.47 \\ \hline 
\end{tabular}
\end{table}


\begin{table}[H]
\centering
\caption{Statsmodels - Insurance Charges Linear Regression}
\label{tab:Statsmodels - Insurance Charges Linear Regression}
\begin{tabular}{|c|c|c|}\hline
Metric & In-Sample & 80-20 Split \\ \hline \hline
rSq & 0.7509 & 0.7836 \\ \hline
rSqBar & 0.7494 & 0.7778 \\ \hline
sst & 196074221568.3671 & 41606660039.7953 \\ \hline
sse & 48839532843.9219 & 9003973448.1649 \\ \hline
sde & 6062.1023 & 5884.7827 \\ \hline
mse0 & 36749084.1564 & 33596915.8514 \\ \hline
rmse & 6062.1023 & 5796.2847 \\ \hline
mae & 4170.8869 & 4181.1945 \\ \hline
smape & 37.8059 & 40.0220 \\ \hline
m & 1338.0000 & 268.0000 \\ \hline
dfr & 8.0000 & 8.0000 \\ \hline
df & 1329.0000 & 260.0000 \\ \hline
fStat & 500.8107 & 117.6800 \\ \hline
aic & 27113.5058 & 4660.4252 \\ \hline
bic & 27160.2962 & 4689.1531 \\ \hline
\end{tabular}
\end{table}

\Cref{fig:Scalation - Insurance reg} shows that plots for the predicted $y$-values vs. the actual $y$-values from scalation, while \Cref{fig:Statsmodels - Insurance reg} show the same from mathstats. Again the results are very similar.


\begin{figure}[H]
    \centering
    \captionsetup{justification=centering}
    \begin{subfigure}[b]{0.45\textwidth}
        \centering
        \includegraphics[width=\textwidth]{Insurance/scalation_reg_In_Sample.png} 
        %\caption{Description of left image}
        %\label{fig:left}
    \end{subfigure}
    \hfill % Adds flexible space between the images
    \begin{subfigure}[b]{0.45\textwidth}
        \centering
        \includegraphics[width=\textwidth]{Insurance/scalation_reg_80_20.png}
        %\caption{Description of right image}
        %\label{fig:right}
    \end{subfigure}
    \caption{Scalation - Insurance Charges Regression\\ Left: In Sample Predictions\\ Right: 80-20 Out of Sample Predictions\\ yy black/actual vs. yp red/predicted}
    \label{fig:Scalation - Insurance reg}
\end{figure}

\begin{figure}[H]
    \centering
    \captionsetup{justification=centering}
    \begin{subfigure}[b]{0.45\textwidth}
        \centering
        \includegraphics[width=1.1\textwidth]{Insurance/statsmodels_reg_In_Sample.png} 
        %\caption{Description of left image}
        %\label{fig:left}
    \end{subfigure}
    \hfill % Adds flexible space between the images
    \begin{subfigure}[b]{0.45\textwidth}
        \centering
        \includegraphics[width=1.1\textwidth]{Insurance/statsmodels_reg_80_20.png}
        %\caption{Description of right image}
        %\label{fig:right}
    \end{subfigure}
    \caption{Statsmodels - Insurance Charges Regression\\ Left: In Sample Predictions\\ Right: 80-20 Out of Sample Predictions\\ yy black/actual vs. yp red/predicted}
    \label{fig:Statsmodels - Insurance reg}
\end{figure}

Next, we performed 5 fold cross validation. \Cref{tab: Scalation - Insurance Charges Linear Regression CV,tab:Statsmodels - Insurance Charges Linear Regression CV} show the resulting quality of fit metrics from scalation and statsmodels respectively. Again we see that the results are similar.

\begin{table}[H]
\centering
\caption{Scalation - Insurance Charges Linear Regression CV}
\label{tab: Scalation - Insurance Charges Linear Regression CV}
\begin{tabular}{|c|c|c|c|c|c|c|} \hline 
Name &  num & min & max & mean & stdev & interval \\ \hline \hline 
rSq & 5 & 0.701 &      0.814 &      0.743 &      0.046 &      0.057 \\ \hline 
rSqBar & 5 & 0.699 &      0.813 &      0.742 &      0.046 &      0.057 \\ \hline 
sst & 5 & 31902777173.848 & 43430486230.657 & 38949749086.422 & 4343029725.651 & 5393640012.108 \\ \hline 
sse & 5 & 7539480548.420 & 11454966761.392 & 9918152392.401 & 1670500799.560 & 2074607019.047 \\ \hline 
sde & 5 & 5320.957 &   6562.018 &   6084.654 &    526.821 &    654.263 \\ \hline 
mse0 & 5 & 28237754.863 & 42902497.234 & 37146638.174 & 6256557.302 & 7770063.742 \\ \hline 
rmse & 5 & 5313.921 &   6550.000 &   6076.688 &    525.006 &    652.009 \\ \hline 
mae & 5 & 3810.754 &   4511.498 &   4216.703 &    292.688 &    363.491 \\ \hline 
smape & 5 & 36.128 &     40.060 &     38.143 &      1.833 &      2.277 \\ \hline 
m & 5 & 267.000 &    267.000 &    267.000 &      0.000 &      0.000 \\ \hline 
dfr & 5 & 7.000 &      7.000 &      7.000 &      0.000 &      0.000 \\ \hline 
df & 5 & 1330.000 &   1330.000 &   1330.000 &     0.000 &      0.000 \\ \hline 
fStat & 5 & 444.882 &    830.519 &    573.015 &    157.534 &    195.643 \\ \hline 
aic & 5 & -2709.047 &  -2663.110 &  -2691.017 &     19.599 &     24.340 \\ \hline 
bic & 5 & -2680.349 &  -2634.412 &  -2662.319 &     19.599 &     24.340 \\ \hline 
\end{tabular}
\end{table}

\begin{table}[H]
\centering
\caption{Statsmodels - Insurance Charges Linear Regression CV}
\label{tab:Statsmodels - Insurance Charges Linear Regression CV}
\begin{tabular}{|c|c|c|c|c|c|}\hline
Name & In-num folds & min & max & mean & stdev \\ \hline \hline
rSq & 5 & 0.6324 & 0.7956 & 0.7402 & 0.0578 \\ \hline
rSqBar & 5 & 0.6225 & 0.7901 & 0.7332 & 0.0593 \\ \hline
sst & 5 & 30189024179.7055 & 43857198758.4016 & 39154186092.5516 & 4823002859.8713 \\ \hline
sse & 5 & 8965018845.2774 & 11096336332.7613 & 9899546225.2180 & 821555419.0107 \\ \hline
sde & 5 & 5872.0390 & 6545.4562 & 6170.1628 & 260.9758 \\ \hline
mse0 & 5 & 33451562.8555 & 41559312.1077 & 36998683.9155 & 3128575.4360 \\ \hline
rmse & 5 & 5783.7326 & 6446.6512 & 6077.2269 & 256.8986 \\ \hline
mae & 5 & 4054.1099 & 4427.9335 & 4203.4121 & 129.0554 \\ \hline
smape & 5 & 35.6194 & 40.0220 & 38.1279 & 1.5723 \\ \hline
m & 5 & 267.0000 & 268.0000 & 267.6000 & 0.4899 \\ \hline
dfr & 5 & 8.0000 & 8.0000 & 8.0000 & 0.0000 \\ \hline
df & 5 & 259.0000 & 260.0000 & 259.6000 & 0.4899 \\ \hline
fStat & 5 & 55.7054 & 126.4912 & 97.8508 & 24.6138 \\ \hline
aic & 5 & 4659.2632 & 4699.8828 & 4678.3124 & 16.0624 \\ \hline
bic & 5 & 4687.9911 & 4728.5808 & 4707.0283 & 16.0528 \\ \hline
\end{tabular}
\end{table}

Finally, we apply forward selection, backward elimination, and stepwise selection to determine which variables best explain response variable. \Cref{fig:Scalation - Insurance reg FS} shows plots for $R^2$, $\overline{R^2}$, sMAPE, $R^2$ cv, and AIC vs. $n$ (the number of variables selected) when utilizing forward selection. From the left plot, it is clear that only $s$ variables are needed to explain the response variable. The order in which the variables were chosen was 1. smoker\_yes, 2. age, 3. bmi, 4. children, 5. region\_southeast, 6. sex\_male, and 7. region\_northwest

\begin{figure}[H]
    \centering
    \captionsetup{justification=centering}
    \begin{subfigure}[b]{0.45\textwidth}
        \centering
        \includegraphics[width=\textwidth]{Insurance/scalation_reg_FS_R2.png} 
        %\caption{Description of left image}
        %\label{fig:left}
    \end{subfigure}
    \hfill % Adds flexible space between the images
    \begin{subfigure}[b]{0.45\textwidth}
        \centering
        \includegraphics[width=\textwidth]{Insurance/scalation_reg_FS_aic.png}
        %\caption{Description of right image}
        %\label{fig:right}
    \end{subfigure}
    \caption{Scalation - Insurance Charges Regression Forward Selection\\ Left: $R^2$ vs. $n$\\ Right: aic vs. $n$}
    \label{fig:Scalation - Insurance reg FS}
\end{figure}

\Cref{fig:Scalation - Insurance reg BE} shows plots for $R^2$, $\overline{R^2}$, sMAPE, $R^2$ cv, and AIC vs. $n$ (the number of variables selected) when utilizing backward elimination. From the left plot, it is again clear that only $2$ variables are needed to explain the response variable. Here, both backward and forward selection agree with each other in their selection of variables, so if you were to remove one variable at a time, in order you would remove 1. region\_northwest, 2. sex\_male, 3. region\_southeast, 4. children, 5. bmi, 6. age, and 7. smoker\_yes. This is the reverse order of forward selection, meaning that they agree.

\begin{figure}[H]
    \centering
    \captionsetup{justification=centering}
    \begin{subfigure}[b]{0.45\textwidth}
        \centering
        \includegraphics[width=\textwidth]{Insurance/scalation_reg_BE_R2.png} 
        %\caption{Description of left image}
        %\label{fig:left}
    \end{subfigure}
    \hfill % Adds flexible space between the images
    \begin{subfigure}[b]{0.45\textwidth}
        \centering
        \includegraphics[width=\textwidth]{Insurance/scalation_reg_BE_aic.png}
        %\caption{Description of right image}
        %\label{fig:right}
    \end{subfigure}
    \caption{Scalation - Insurance Charges Regression Backward Elimination\\ Left: $R^2$ vs. $n$\\ Right: aic vs. $n$}
    \label{fig:Scalation - Insurance reg BE}
\end{figure}

Last, \Cref{fig:Scalation - Insurance reg SS} shows plots for $R^2$, $\overline{R^2}$, sMAPE, $R^2$ cv, and AIC vs. $n$ (the number of variables selected) when utilizing stepwise selection. We still find that if we want to get the best models recommended from stepwise, we should add in the following order, 1. smoker\_yes, 2. age, 3. bmi, 4. children, 5. region\_southeast, and 6. sex\_male. Note that this almost agrees with forward selection and backward elimination, but we do not add region\_northwest as stepwise decided that moving was not worthwhile.

\begin{figure}[H]
    \centering
    \captionsetup{justification=centering}
    \begin{subfigure}[b]{0.45\textwidth}
        \centering
        \includegraphics[width=\textwidth]{Insurance/scalation_reg_SS_R2.png} 
        %\caption{Description of left image}
        %\label{fig:left}
    \end{subfigure}
    \hfill % Adds flexible space between the images
    \begin{subfigure}[b]{0.45\textwidth}
        \centering
        \includegraphics[width=\textwidth]{Insurance/scalation_reg_SS_aic.png}
        %\caption{Description of right image}
        %\label{fig:right}
    \end{subfigure}
    \caption{Scalation - Insurance Charges Regression Stepwise Selection\\ Left: $R^2$ vs. $n$\\ Right: aic vs. $n$}
    \label{fig:Scalation - Insurance reg SS}
\end{figure}




























\section{Ridge Regression}

\subsection{Auto MPG}

\begin{figure}[h]
    \centering
    \captionsetup{justification=centering}
    \begin{subfigure}[b]{0.45\textwidth}
        \centering
        \includegraphics[width=\textwidth]{Auto_MPG/scalation_ridge_In_Sample.png} 
        %\caption{Description of left image}
        %\label{fig:left}
    \end{subfigure}
    \hfill % Adds flexible space between the images
    \begin{subfigure}[b]{0.45\textwidth}
        \centering
        \includegraphics[width=\textwidth]{Auto_MPG/scalation_ridge_80_20.png}
        %\caption{Description of right image}
        %\label{fig:right}
    \end{subfigure}
    \caption{Scalation - Auto MPG Ridge\\ Left: In Sample Predictions\\ Right: 80-20 Out of Sample Predictions\\ yy black/actual vs. yp red/predicted}
    \label{fig:Scalation - AutoMPG ridge}
\end{figure}

\begin{figure}[h]
    \centering
    \captionsetup{justification=centering}
    \begin{subfigure}[b]{0.45\textwidth}
        \centering
        \includegraphics[width=1.1\textwidth]{Auto_MPG/statsmodels_ridge_In_Sample.png} 
        %\caption{Description of left image}
        %\label{fig:left}
    \end{subfigure}
    \hfill % Adds flexible space between the images
    \begin{subfigure}[b]{0.45\textwidth}
        \centering
        \includegraphics[width=1.1\textwidth]{Auto_MPG/statsmodels_ridge_80_20.png}
        %\caption{Description of right image}
        %\label{fig:right}
    \end{subfigure}
    \caption{Statsmodels - Auto MPG Ridge\\ Left: In Sample Predictions\\ Right: 80-20 Out of Sample Predictions\\ yy black/actual vs. yp red/predicted}
    \label{fig:Statsmodels - AutoMPG ridge}
\end{figure}

\subsection{House Prices}


\begin{figure}[h]
    \centering
    \captionsetup{justification=centering}
    \begin{subfigure}[b]{0.45\textwidth}
        \centering
        \includegraphics[width=\textwidth]{Housing/scalation_ridge_In_Sample.png} 
        %\caption{Description of left image}
        %\label{fig:left}
    \end{subfigure}
    \hfill % Adds flexible space between the images
    \begin{subfigure}[b]{0.45\textwidth}
        \centering
        \includegraphics[width=\textwidth]{Housing/scalation_ridge_80_20.png}
        %\caption{Description of right image}
        %\label{fig:right}
    \end{subfigure}
    \caption{Scalation - House Price Ridge\\ Left: In Sample Predictions\\ Right: 80-20 Out of Sample Predictions\\ yy black/actual vs. yp red/predicted}
    \label{fig:Scalation - Housing ridge}
\end{figure}

\begin{figure}[h]
    \centering
    \captionsetup{justification=centering}
    \begin{subfigure}[b]{0.45\textwidth}
        \centering
        \includegraphics[width=1.1\textwidth]{Housing/statsmodels_ridge_In_Sample.png} 
        %\caption{Description of left image}
        %\label{fig:left}
    \end{subfigure}
    \hfill % Adds flexible space between the images
    \begin{subfigure}[b]{0.45\textwidth}
        \centering
        \includegraphics[width=1.1\textwidth]{Housing/statsmodels_ridge_80_20.png}
        %\caption{Description of right image}
        %\label{fig:right}
    \end{subfigure}
    \caption{Statsmodels - House Price Ridge\\ Left: In Sample Predictions\\ Right: 80-20 Out of Sample Predictions\\ yy black/actual vs. yp red/predicted}
    \label{fig:Statsmodels - Housing ridge}
\end{figure}


\subsection{Insurance Charges}


\begin{figure}[h]
    \centering
    \captionsetup{justification=centering}
    \begin{subfigure}[b]{0.45\textwidth}
        \centering
        \includegraphics[width=\textwidth]{Insurance/scalation_ridge_In_Sample.png} 
        %\caption{Description of left image}
        %\label{fig:left}
    \end{subfigure}
    \hfill % Adds flexible space between the images
    \begin{subfigure}[b]{0.45\textwidth}
        \centering
        \includegraphics[width=\textwidth]{Insurance/scalation_ridge_80_20.png}
        %\caption{Description of right image}
        %\label{fig:right}
    \end{subfigure}
    \caption{Scalation - Insurance Charges Ridge\\ Left: In Sample Predictions\\ Right: 80-20 Out of Sample Predictions\\ yy black/actual vs. yp red/predicted}
    \label{fig:Scalation - Insurance ridge}
\end{figure}

\begin{figure}[h]
    \centering
    \captionsetup{justification=centering}
    \begin{subfigure}[b]{0.45\textwidth}
        \centering
        \includegraphics[width=1.1\textwidth]{Insurance/statsmodels_ridge_In_Sample.png} 
        %\caption{Description of left image}
        %\label{fig:left}
    \end{subfigure}
    \hfill % Adds flexible space between the images
    \begin{subfigure}[b]{0.45\textwidth}
        \centering
        \includegraphics[width=1.1\textwidth]{Insurance/statsmodels_ridge_80_20.png}
        %\caption{Description of right image}
        %\label{fig:right}
    \end{subfigure}
    \caption{Statsmodels - Insurance Charges Ridge\\ Left: In Sample Predictions\\ Right: 80-20 Out of Sample Predictions\\ yy black/actual vs. yp red/predicted}
    \label{fig:Statsmodels - Insurance ridge}
\end{figure}



















\section*{Transformed Regression: Sqrt, Log1p, Box-Cox}
In this section, we applied three different transformation techniques to the response variables (mpg, house price, and insurance charge) across the Auto MPG, Boston House Price, and Medical Cost datasets. The distributions for mpg, house price, and medical charge exhibit right-skewed patterns. Therefore, these transformations could significantly improved model accuracy. To measure and compare the quality of fit for the different transformation models, we extracted 15 metrics, including $R^{2}$, adjusted $R^{2}$, MSE, and MAE. Each model was evaluated using both in-sample validation and a validation set with an 80-20\% split. Finally, we applied feature selection to identify the optimal set of features that best describe the response variable. For box-cox transformation we reqired a optimal lamda parameter which was 0.19, 0.85, and 0.04 for mpg, house price, and insurance cost respectively. 

\subsection*{Transformed Regression using Scala}
We used TranRegression function form the modeling to perfrom Sqrt, Log1p, Box-Cox on the Auto MPG, Boston House Price, and Medical Cost datasets. The function takes predictor variables, a response variable, feature names, a factorization method, and transformation and inverse transformation methods as input, and fits the model based on the specified transformation. An example of the code used to fit the square-root (sqrt) transformation model is provided below.

\begin{verbatim}
f = ("sqrt, sq, "sqrt")
val mod = new modeling.TranRegression (ox, y, ox_fname, 
                                       modeling.Regression.hp, 
                                       f._1, f._2, f._3)  
\end{verbatim}

\subsection*{Transformed Regression on the Auto MPG Dataset using Scala: In-Sample and Validation Results}
Tables 1 and 2 present the quality of fit for the in-sample and validation (80-20\% split) evaluations using the Auto MPG data. For the in-sample case, the models utilized all available data. The results indicate that the log1p and Box-Cox transformations outperform the sqrt transformation, which is further confirmed by the adjusted $R^{2}$ values. The Mean Square Error (MSE) for the sqrt, log1p, and Box-Cox transformations reached 10.02, 9.17, and 9.40, respectively, with the log1p transformation achieving the lowest error. Similarly, in the validation accuracy results, where 80\% of the data was randomly used for model training and 20\% for testing, the log1p transformation achieved the highest $R^{2}$ and adjusted $R^{2}$ values, as well as the lowest MSE.

\begin{table}[h]
\centering
\caption{Auto MPG (In-Sample): Sqrt, Log1p, and Box–Cox($\lambda$ = 0.19)}
\label{tab:Auto_MPG_(In-Sample)}
\begin{tabular}{|c|c|c|c|} \hline 
Metric & sqrt & log1p & box-cox($\lambda$=0.19) \\ \hline \hline 
rSq &0.835138 & 0.849102 &      0.845366 \\ \hline 
rSqBar &0.832569 &      0.846751 &      0.842956 \\ \hline 
sst &23819.0 &  23819.0 &       23819.0 \\ \hline 
sse &3926.85 &  3594.23 &       3683.22 \\ \hline 
sde &3.16757 &  3.02514 &       3.06455 \\ \hline 
mse0 &10.0175 & 9.16895 &       9.39596 \\ \hline 
rmse &3.16504 & 3.02803 &       3.06528 \\ \hline 
mae &2.34675 &  2.18422 &       2.22945 \\ \hline 
smape &10.2046 &        9.32001 &       9.54480 \\ \hline 
m &392.000 &    392.000 &       392.000 \\ \hline 
dfr &6.00000 &  6.00000 &       6.00000 \\ \hline 
df &385.000 &   385.000 &       385.000 \\ \hline 
fStat &325.048 &        361.067 &       350.793 \\ \hline 
aic &-993.873 & -976.527 &      -981.320 \\ \hline 
bic &-966.074 & -948.728 &      -953.521 \\ \hline 
\end{tabular}
\end{table}

\begin{table}[h]
\centering
\caption{Auto MPG (Validation): Sqrt, Log1p, and Box–Cox($\lambda$ = 0.19)}
\label{tab:Auto_MPG_(Validation)}
\begin{tabular}{|c|c|c|c|} \hline 
Metric & sqrt & log1p & box-cox($\lambda$=0.19) \\ \hline \hline 
rSq &0.846480 & 0.852864 &      0.851819 \\ \hline 
rSqBar &0.844088 &      0.850571 &      0.849510 \\ \hline 
sst &4731.23 &  4731.23 &       4731.23 \\ \hline 
sse &726.337 &  696.133 &       701.080 \\ \hline 
sde &3.05724 &  2.97969 &       2.99552 \\ \hline 
mse0 &9.31202 & 8.92478 &       8.98820 \\ \hline 
rmse &3.05156 & 2.98744 &       2.99803 \\ \hline 
mae &2.11846 &  1.98665 &       2.01582 \\ \hline 
smape &9.00068 &        8.28831 &       8.41880 \\ \hline 
m &78.0000 &    78.0000 &       78.0000 \\ \hline 
dfr &6.00000 &  6.00000 &       6.00000 \\ \hline 
df &385.000 &   385.000 &       385.000 \\ \hline 
fStat &353.804 &        371.939 &       368.862 \\ \hline 
aic &-183.700 & -182.048 &      -182.322 \\ \hline 
bic &-167.203 & -165.551 &      -165.825 \\ \hline 
\end{tabular}
\end{table}

\begin{figure}[h]
    \centering
    \captionsetup{justification=centering}
    \begin{subfigure}[b]{0.45\textwidth}
        \centering
        \includegraphics[width=\textwidth]{Auto_MPG/scalation_sqrt_In_Sample.png} 
        %\caption{Description of left image}
        %\label{fig:left}
    \end{subfigure}
    \hfill % Adds flexible space between the images
    \begin{subfigure}[b]{0.45\textwidth}
        \centering
        \includegraphics[width=\textwidth]{Auto_MPG/scalation_sqrt_80_20.png}
        %\caption{Description of right image}
        %\label{fig:right}
    \end{subfigure}
    \caption{Scalation - Auto MPG Sqrt\\ Left: In Sample Predictions\\ Right: 80-20 Out of Sample Predictions\\ yy black/actual vs. yp red/predicted}
    \label{fig:Scalation - AutoMPG sqrt}
\end{figure}

\begin{figure}[h]
    \centering
    \captionsetup{justification=centering}
    \begin{subfigure}[b]{0.45\textwidth}
        \centering
        \includegraphics[width=1.1\textwidth]{Auto_MPG/statsmodels_sqrt_In_Sample.png} 
        %\caption{Description of left image}
        %\label{fig:left}
    \end{subfigure}
    \hfill % Adds flexible space between the images
    \begin{subfigure}[b]{0.45\textwidth}
        \centering
        \includegraphics[width=1.1\textwidth]{Auto_MPG/statsmodels_sqrt_80_20.png}
        %\caption{Description of right image}
        %\label{fig:right}
    \end{subfigure}
    \caption{Statsmodels - Auto MPG Sqrt\\ Left: In Sample Predictions\\ Right: 80-20 Out of Sample Predictions\\ yy black/actual vs. yp red/predicted}
    \label{fig:Statsmodels - AutoMPG sqrt}
\end{figure}

\begin{figure}[h]
    \centering
    \captionsetup{justification=centering}
    \begin{subfigure}[b]{0.45\textwidth}
        \centering
        \includegraphics[width=\textwidth]{Auto_MPG/scalation_log1p_In_Sample.png} 
        %\caption{Description of left image}
        %\label{fig:left}
    \end{subfigure}
    \hfill % Adds flexible space between the images
    \begin{subfigure}[b]{0.45\textwidth}
        \centering
        \includegraphics[width=\textwidth]{Auto_MPG/scalation_log1p_80_20.png}
        %\caption{Description of right image}
        %\label{fig:right}
    \end{subfigure}
    \caption{Scalation - Auto MPG log1p\\ Left: In Sample Predictions\\ Right: 80-20 Out of Sample Predictions\\ yy black/actual vs. yp red/predicted}
    \label{fig:Scalation - AutoMPG log1p}
\end{figure}

\begin{figure}[h]
    \centering
    \captionsetup{justification=centering}
    \begin{subfigure}[b]{0.45\textwidth}
        \centering
        \includegraphics[width=1.1\textwidth]{Auto_MPG/statsmodels_log1p_In_Sample.png} 
        %\caption{Description of left image}
        %\label{fig:left}
    \end{subfigure}
    \hfill % Adds flexible space between the images
    \begin{subfigure}[b]{0.45\textwidth}
        \centering
        \includegraphics[width=1.1\textwidth]{Auto_MPG/statsmodels_log1p_80_20.png}
        %\caption{Description of right image}
        %\label{fig:right}
    \end{subfigure}
    \caption{Statsmodels - Auto MPG log1p\\ Left: In Sample Predictions\\ Right: 80-20 Out of Sample Predictions\\ yy black/actual vs. yp red/predicted}
    \label{fig:Statsmodels - AutoMPG log1p}
\end{figure}

\subsection*{Transformed Regression on the Boston House Price Dataset: In-Sample and Validation Results}
For the house price prediction, the log1p transformed model performed significantly worse than the sqrt and Box-Cox transformations. The Box-Cox and sqrt models performed similarly, although Box-Cox achieved highest $R^{2}$ and adjusted $R^{2}$ of 0.998 in the in-sample evaluation. A similar trend was observed in the validation, where Box-Cox and sqrt outperformed the log1p transformation, with Box-Cox achieving the highest $R^{2}$ and adjusted $R^{2}$.

\begin{table}[h]
\centering
\caption{House Price (In-Sample): Sqrt, Log1p, and Box–Cox($\lambda$ = 0.85)}
\label{tab:House_Price_(In-Sample)}
\begin{tabular}{|c|c|c|c|} \hline 
Metric & sqrt & log1p & box-cox($\lambda$=0.85) \\ \hline \hline 
rSq &0.986164 & 0.922376 &      0.997549 \\ \hline 
rSqBar &0.986067 &      0.921828 &      0.997531 \\ \hline 
sst &6.42325e+13 &      6.42325e+13 &   6.42325e+13 \\ \hline 
sse &8.88711e+11 &      4.98598e+12 &   1.57458e+11 \\ \hline 
sde &29823.1 &  70573.3 &       12554.0 \\ \hline 
mse0 &8.88711e+08 &     4.98598e+09 &   1.57458e+08 \\ \hline 
rmse &29811.3 & 70611.5 &       12548.2 \\ \hline 
mae &24296.5 &  52989.3 &       10078.9 \\ \hline 
smape &4.85788 &        9.21218 &       2.12365 \\ \hline 
m &1000.00 &    1000.00 &       1000.00 \\ \hline 
dfr &7.00000 &  7.00000 &       7.00000 \\ \hline 
df &992.000 &   992.000 &       992.000 \\ \hline 
fStat &10100.8 &        1683.94 &       57668.5 \\ \hline 
aic &-11705.6 & -12567.9 &      -10840.3 \\ \hline 
bic &-11666.3 & -12528.6 &      -10801.0 \\ \hline 
\end{tabular}
\end{table}

\begin{table}[h]
\centering
\caption{House Price (Validation): Sqrt, Log1p, and Box–Cox($\lambda$ = 0.85)}
\label{tab:House_Price_(Validation)}
\begin{tabular}{|c|c|c|c|} \hline 
Metric & sqrt & log1p & box-cox($\lambda$=0.85) \\ \hline \hline 
rSq &0.984017 & 0.906459 &      0.997398 \\ \hline 
rSqBar &0.983904 &      0.905799 &      0.997380 \\ \hline 
sst &1.33700e+13 &      1.33700e+13 &   1.33700e+13 \\ \hline 
sse &2.13694e+11 &      1.25064e+12 &   3.47839e+10 \\ \hline 
sde &32755.5 &  78812.3 &       13217.8 \\ \hline 
mse0 &1.06847e+09 &     6.25321e+09 &   1.73920e+08 \\ \hline 
rmse &32687.5 & 79077.2 &       13187.9 \\ \hline 
mae &26140.1 &  58024.9 &       10655.9 \\ \hline 
smape &5.18595 &        9.88665 &       2.22264 \\ \hline 
m &200.000 &    200.000 &       200.000 \\ \hline 
dfr &7.00000 &  7.00000 &       7.00000 \\ \hline 
df &992.000 &   992.000 &       992.000 \\ \hline 
fStat &8724.77 &        1373.28 &       54329.4 \\ \hline 
aic &-2346.74 & -2523.43 &      -2165.20 \\ \hline 
bic &-2320.35 & -2497.04 &      -2138.81 \\ \hline 
\end{tabular}
\end{table}

\begin{figure}[h]
    \centering
    \captionsetup{justification=centering}
    \begin{subfigure}[b]{0.45\textwidth}
        \centering
        \includegraphics[width=\textwidth]{Housing/scalation_sqrt_In_Sample.png} 
        %\caption{Description of left image}
        %\label{fig:left}
    \end{subfigure}
    \hfill % Adds flexible space between the images
    \begin{subfigure}[b]{0.45\textwidth}
        \centering
        \includegraphics[width=\textwidth]{Housing/scalation_sqrt_80_20.png}
        %\caption{Description of right image}
        %\label{fig:right}
    \end{subfigure}
    \caption{Scalation - House Price Sqrt\\ Left: In Sample Predictions\\ Right: 80-20 Out of Sample Predictions\\ yy black/actual vs. yp red/predicted}
    \label{fig:Scalation - Housing sqrt}
\end{figure}

\begin{figure}[h]
    \centering
    \captionsetup{justification=centering}
    \begin{subfigure}[b]{0.45\textwidth}
        \centering
        \includegraphics[width=1.1\textwidth]{Housing/statsmodels_sqrt_In_Sample.png} 
        %\caption{Description of left image}
        %\label{fig:left}
    \end{subfigure}
    \hfill % Adds flexible space between the images
    \begin{subfigure}[b]{0.45\textwidth}
        \centering
        \includegraphics[width=1.1\textwidth]{Housing/statsmodels_sqrt_80_20.png}
        %\caption{Description of right image}
        %\label{fig:right}
    \end{subfigure}
    \caption{Statsmodels - House Price Sqrt\\ Left: In Sample Predictions\\ Right: 80-20 Out of Sample Predictions\\ yy black/actual vs. yp red/predicted}
    \label{fig:Statsmodels - Housing sqrt}
\end{figure}

\begin{figure}[h]
    \centering
    \captionsetup{justification=centering}
    \begin{subfigure}[b]{0.45\textwidth}
        \centering
        \includegraphics[width=\textwidth]{Housing/scalation_log1p_In_Sample.png} 
        %\caption{Description of left image}
        %\label{fig:left}
    \end{subfigure}
    \hfill % Adds flexible space between the images
    \begin{subfigure}[b]{0.45\textwidth}
        \centering
        \includegraphics[width=\textwidth]{Housing/scalation_log1p_80_20.png}
        %\caption{Description of right image}
        %\label{fig:right}
    \end{subfigure}
    \caption{Scalation - House Price log1p\\ Left: In Sample Predictions\\ Right: 80-20 Out of Sample Predictions\\ yy black/actual vs. yp red/predicted}
    \label{fig:Scalation - Housing log1p}
\end{figure}

\begin{figure}[h]
    \centering
    \captionsetup{justification=centering}
    \begin{subfigure}[b]{0.45\textwidth}
        \centering
        \includegraphics[width=1.1\textwidth]{Housing/statsmodels_log1p_In_Sample.png} 
        %\caption{Description of left image}
        %\label{fig:left}
    \end{subfigure}
    \hfill % Adds flexible space between the images
    \begin{subfigure}[b]{0.45\textwidth}
        \centering
        \includegraphics[width=1.1\textwidth]{Housing/statsmodels_log1p_80_20.png}
        %\caption{Description of right image}
        %\label{fig:right}
    \end{subfigure}
    \caption{Statsmodels - House Price log1p\\ Left: In Sample Predictions\\ Right: 80-20 Out of Sample Predictions\\ yy black/actual vs. yp red/predicted}
    \label{fig:Statsmodels - Housing log1p}
\end{figure}

\subsection*{Transformed Regression on the Medical Cost Dataset: In-Sample and Validation Results}
In the Medical Cost dataset, the sqrt transformation performed significantly better than the log1p and Box-Cox transformations. The sqrt transformation achieved an $R^{2}$ of 0.753 and an adjusted $R^{2}$ of 0.751 in the in-sample evaluation, compared to 0.730 and 0.729, respectively, for the validation set. The adjusted $R^{2}$ also shows a similar trend, where sqrt, log1p, and Box-Cox achieved 0.751, 0.520, and 0.574 for in-sample evaluation and 0.729, 0.569, and 0.607 for validation, respectively.

\begin{table}[h]
\centering
\caption{Medical Cost (In-Sample): Sqrt, Log1p, and Box–Cox($\lambda$ = 0.04)}
\label{tab:Medical_Cost_(In-Sample)}
\begin{tabular}{|c|c|c|c|} \hline 
Metric & sqrt & log1p & box-cox($\lambda$=0.04) \\ \hline \hline 
rSq &0.752657 & 0.522782 &      0.576265 \\ \hline 
rSqBar &0.751168 &      0.519909 &      0.573715 \\ \hline 
sst &1.96074e+11 &      1.96074e+11 &   1.96074e+11 \\ \hline 
sse &4.84975e+10 &      9.35702e+10 &   8.30835e+10 \\ \hline 
sde &6001.71 &  8358.44 &       7870.39 \\ \hline 
mse0 &3.62463e+07 &     6.99329e+07 &   6.20953e+07 \\ \hline 
rmse &6020.49 & 8362.59 &       7880.06 \\ \hline 
mae &3613.90 &  4219.51 &       4052.90 \\ \hline 
smape &27.6903 &        26.2889 &       26.0851 \\ \hline 
m &1338.00 &    1338.00 &       1338.00 \\ \hline 
dfr &8.00000 &  8.00000 &       8.00000 \\ \hline 
df &1329.00 &   1329.00 &       1329.00 \\ \hline 
fStat &505.514 &        181.986 &       225.924 \\ \hline 
aic &-13525.1 & -13964.7 &      -13885.2 \\ \hline 
bic &-13478.3 & -13917.9 &      -13838.4 \\ \hline 
\end{tabular}
\end{table}

\begin{table}[h]
\centering
\caption{Medical Cost (Validation): Sqrt, Log1p, and Box–Cox($\lambda$ = 0.04)}
\label{tab:Medical_Cost_(Validation)}
\begin{tabular}{|c|c|c|c|} \hline 
Metric & sqrt & log1p & box-cox($\lambda$=0.04) \\ \hline \hline 
rSq &0.730154 & 0.571048 &      0.609124 \\ \hline 
rSqBar &0.728530 &      0.568466 &      0.606771 \\ \hline 
sst &4.06432e+10 &      4.06432e+10 &   4.06432e+10 \\ \hline 
sse &1.09674e+10 &      1.74340e+10 &   1.58865e+10 \\ \hline 
sde &6405.50 &  8088.13 &       7716.02 \\ \hline 
mse0 &4.10764e+07 &     6.52957e+07 &   5.94998e+07 \\ \hline 
rmse &6409.08 & 8080.58 &       7713.61 \\ \hline 
mae &3839.79 &  4116.83 &       3998.83 \\ \hline 
smape &30.1151 &        27.3599 &       27.2984 \\ \hline 
m &267.000 &    267.000 &       267.000 \\ \hline 
dfr &8.00000 &  8.00000 &       8.00000 \\ \hline 
df &1329.00 &   1329.00 &       1329.00 \\ \hline 
fStat &449.505 &        221.156 &       258.882 \\ \hline 
aic &-2701.24 & -2763.11 &      -2750.71 \\ \hline 
bic &-2668.95 & -2730.83 &      -2718.42 \\ \hline 
\end{tabular}
\end{table}

\begin{figure}[h]
    \centering
    \captionsetup{justification=centering}
    \begin{subfigure}[b]{0.45\textwidth}
        \centering
        \includegraphics[width=\textwidth]{Insurance/scalation_sqrt_In_Sample.png} 
        %\caption{Description of left image}
        %\label{fig:left}
    \end{subfigure}
    \hfill % Adds flexible space between the images
    \begin{subfigure}[b]{0.45\textwidth}
        \centering
        \includegraphics[width=\textwidth]{Insurance/scalation_sqrt_80_20.png}
        %\caption{Description of right image}
        %\label{fig:right}
    \end{subfigure}
    \caption{Scalation - Insurance Charges Sqrt\\ Left: In Sample Predictions\\ Right: 80-20 Out of Sample Predictions\\ yy black/actual vs. yp red/predicted}
    \label{fig:Scalation - Insurance sqrt}
\end{figure}

\begin{figure}[h]
    \centering
    \captionsetup{justification=centering}
    \begin{subfigure}[b]{0.45\textwidth}
        \centering
        \includegraphics[width=1.1\textwidth]{Insurance/statsmodels_sqrt_In_Sample.png} 
        %\caption{Description of left image}
        %\label{fig:left}
    \end{subfigure}
    \hfill % Adds flexible space between the images
    \begin{subfigure}[b]{0.45\textwidth}
        \centering
        \includegraphics[width=1.1\textwidth]{Insurance/statsmodels_sqrt_80_20.png}
        %\caption{Description of right image}
        %\label{fig:right}
    \end{subfigure}
    \caption{Statsmodels - Insurance Charges Sqrt\\ Left: In Sample Predictions\\ Right: 80-20 Out of Sample Predictions\\ yy black/actual vs. yp red/predicted}
    \label{fig:Statsmodels - Insurance sqrt}
\end{figure}

\begin{figure}[h]
    \centering
    \captionsetup{justification=centering}
    \begin{subfigure}[b]{0.45\textwidth}
        \centering
        \includegraphics[width=\textwidth]{Insurance/scalation_log1p_In_Sample.png} 
        %\caption{Description of left image}
        %\label{fig:left}
    \end{subfigure}
    \hfill % Adds flexible space between the images
    \begin{subfigure}[b]{0.45\textwidth}
        \centering
        \includegraphics[width=\textwidth]{Insurance/scalation_log1p_80_20.png}
        %\caption{Description of right image}
        %\label{fig:right}
    \end{subfigure}
    \caption{Scalation - Insurance Charges log1p\\ Left: In Sample Predictions\\ Right: 80-20 Out of Sample Predictions\\ yy black/actual vs. yp red/predicted}
    \label{fig:Scalation - Insurance log1p}
\end{figure}

\begin{figure}[h]
    \centering
    \captionsetup{justification=centering}
    \begin{subfigure}[b]{0.45\textwidth}
        \centering
        \includegraphics[width=1.1\textwidth]{Insurance/statsmodels_log1p_In_Sample.png} 
        %\caption{Description of left image}
        %\label{fig:left}
    \end{subfigure}
    \hfill % Adds flexible space between the images
    \begin{subfigure}[b]{0.45\textwidth}
        \centering
        \includegraphics[width=1.1\textwidth]{Insurance/statsmodels_log1p_80_20.png}
        %\caption{Description of right image}
        %\label{fig:right}
    \end{subfigure}
    \caption{Statsmodels - Insurance Charges log1p\\ Left: In Sample Predictions\\ Right: 80-20 Out of Sample Predictions\\ yy black/actual vs. yp red/predicted}
    \label{fig:Statsmodels - Insurance log1p}
\end{figure}

\section*{Transformed Regression using statsmodels}
Here, we used Python statsmodels library to reproduce all results generated by the TranRegression package.


%%%%%%%%%%%%%%%%%% statsmodels %%%%%%%%%%%%%%%%%%%%%%%%%%%%%%%%


%%%%%%%%%%%%%%%%%% Auto MPG %%%%%%%%%%%%%%%%%%%%%%%%%%%%%%%%%%%
\subsection*{Transformed Regression with Sqrt, Log1p, and Box--Cox Transformations on the Auto MPG Dataset: In-Sample, Validation, Forward, and Backward Results}

The following three tables present the results for the sqrt, log1p, and Box-Cox transformations, including in-sample evaluation, validation, and forward and backward feature selection. From these tables, we can see that, similar to the Scala results, the log1p and Box-Cox transformations perform similarly, with log1p being slightly better than Box-Cox. For the in-sample case, the sqrt, log1p, and Box-Cox transformations achieved $R^{2}$ values of 0.835, 0.849, and 0.845, respectively, with adjusted $R^{2}$ values of 0.833, 0.847, and 0.843. Similarly, for the validation case, the $R^{2}$ values were 0.822, 0.832, and 0.830, with adjusted $R^{2}$ values of 0.807, 0.818, and 0.816.
\begin{table}[h]
\centering
\caption{Auto MPG Regression with Square-Root Transformation}
\label{tab:auto_mpg_sqrt}
\begin{tabular}{lrrrr}
\toprule
 & In-Sample & Validation & Forward & Backward \\
\midrule
rSq & 0.835 & 0.822 & 0.823 & 0.822 \\
rSqBar & 0.833 & 0.807 & 0.818 & 0.814 \\
sst & 23818.993 & 5372.801 & 5372.801 & 5372.801 \\
sse & 3926.849 & 955.390 & 950.390 & 958.806 \\
mse & 4.017 & 6.094 & 10.030 & 9.137 \\
rmse & 3.165 & 3.478 & 3.468 & 3.484 \\
mae & 2.347 & 2.376 & 2.375 & 2.350 \\
m & 392.000 & 79.000 & 79.000 & 79.000 \\
dfr & 6.000 & 6.000 & 2.000 & 3.000 \\
df & 385.000 & 72.000 & 76.000 & 75.000 \\
fStat & 825.235 & 120.822 & 220.454 & 161.034 \\
\bottomrule
\end{tabular}

\end{table}

\begin{table}[h]
\centering
\caption{Auto MPG Regression with Log1p Transformation}
\label{tab:auto_mpg_Log1p}
\begin{tabular}{lrrrr}
\toprule
 & In-Sample & Validation & Forward & Backward \\
\midrule
rSq & 0.849 & 0.832 & 0.806 & 0.804 \\
rSqBar & 0.847 & 0.818 & 0.801 & 0.794 \\
sst & 23818.993 & 5372.801 & 5372.801 & 5372.801 \\
sse & 3594.229 & 900.453 & 1044.122 & 1052.182 \\
mse & 3.169 & 5.398 & 11.217 & 9.319 \\
rmse & 3.028 & 3.376 & 3.635 & 3.649 \\
mae & 2.184 & 2.221 & 2.710 & 2.713 \\
m & 392.000 & 79.000 & 79.000 & 79.000 \\
dfr & 6.000 & 6.000 & 2.000 & 4.000 \\
df & 385.000 & 72.000 & 76.000 & 74.000 \\
fStat & 1063.694 & 138.083 & 192.956 & 115.912 \\
\bottomrule
\end{tabular}

\end{table}

\begin{table}[h]
\centering
\caption{Auto MPG Regression with Box-Cox Transformation}
\label{tab:auto_mpg_box_cox}
\begin{tabular}{lrrrr}
\toprule
 & In-Sample & Validation & Forward & Backward \\
\midrule
rSq & 0.845 & 0.830 & 0.829 & 0.829 \\
rSqBar & 0.843 & 0.816 & 0.825 & 0.823 \\
sst & 23818.993 & 5372.801 & 5372.801 & 5372.801 \\
sse & 3683.218 & 914.477 & 917.735 & 916.540 \\
mse & 3.396 & 5.576 & 9.617 & 8.602 \\
rmse & 3.065 & 3.402 & 3.408 & 3.406 \\
mae & 2.229 & 2.262 & 2.356 & 2.234 \\
m & 392.000 & 79.000 & 79.000 & 79.000 \\
dfr & 6.000 & 6.000 & 2.000 & 3.000 \\
df & 385.000 & 72.000 & 76.000 & 75.000 \\
fStat & 988.221 & 133.268 & 231.627 & 172.688 \\
\bottomrule
\end{tabular}

\end{table}

%%%%%%%%%%%%%%%%%% House Price %%%%%%%%%%%%%%%%%%%%%%%%%%%%%%

\subsection*{Transformed Regression with Sqrt, Log1p, and Box--Cox Transformations on the Boston House Price Dataset: In-Sample, Validation, Forward, and Backward Results}

For the house price prediction dataset, the ScalaTion and statsmodels summaries are consistent, as the Box–Cox transformation was identified as the best-performing model under both approaches.


\begin{table}[h]
\centering
\caption{Boston House Price Regression with Square-Root Transformation}
\label{tab:house_price_sqrt}
\begin{tabular}{lrrrr}
\toprule
 & In-Sample & Validation & Forward & Backward \\
\midrule
rSq & 0.986 & 0.987 & 0.281 & 0.981 \\
rSqBar & 0.986 & 0.986 & 0.277 & 0.981 \\
sst & 64232463468052.547 & 11927062037792.469 & 11927062037792.469 & 11927062037792.469 \\
sse & 888710772890.930 & 155868297872.986 & 8579357295076.382 & 223122191452.058 \\
mse & 888710765.891 & 779341482.365 & 42896786474.382 & 1115610953.260 \\
rmse & 29811.252 & 27916.688 & 207115.394 & 33400.763 \\
mae & 24296.468 & 22632.355 & 177100.990 & 27022.336 \\
m & 1000.000 & 200.000 & 200.000 & 200.000 \\
dfr & 7.000 & 7.000 & 1.000 & 4.000 \\
df & 992.000 & 192.000 & 198.000 & 195.000 \\
fStat & 10182.286 & 2157.718 & 78.041 & 2622.765 \\
\bottomrule
\end{tabular}

\end{table}

\begin{table}[h]
\centering
\caption{Boston House Price Regression with Log1p Transformation}
\label{tab:house_price_log1p}
\begin{tabular}{lrrrr}
\toprule
 & In-Sample & Validation & Forward & Backward \\
\midrule
rSq & 0.922 & 0.934 & -81398.531 & 0.877 \\
rSqBar & 0.922 & 0.931 & -81809.639 & 0.875 \\
sst & 64232463468052.547 & 11927062037792.469 & 11927062037792.469 & 11927062037792.469 \\
sse & 4985983480190.929 & 789371506083.466 & 970857251871500928.000 & 1461933128041.826 \\
mse & 4985983473.191 & 3946857523.417 & 4854286259357504.000 & 7309665636.209 \\
rmse & 70611.497 & 62824.020 & 69672708.139 & 85496.583 \\
mae & 52989.293 & 48059.016 & 20766229.670 & 61057.028 \\
m & 1000.000 & 200.000 & 200.000 & 200.000 \\
dfr & 7.000 & 7.000 & 1.000 & 4.000 \\
df & 992.000 & 192.000 & 198.000 & 195.000 \\
fStat & 1697.515 & 403.131 & -199.998 & 357.921 \\
\bottomrule
\end{tabular}

\end{table}

\begin{table}[h]
\centering
\caption{Boston House Price Regression with Box-Cox Transformation}
\label{tab:house_price_box-cox}
\begin{tabular}{lrrrr}
\toprule
 & In-Sample & Validation & Forward & Backward \\
\midrule
rSq & 0.998 & 0.997 & 0.924 & 0.990 \\
rSqBar & 0.998 & 0.997 & 0.924 & 0.989 \\
sst & 64232463468052.531 & 11927062037792.469 & 11927062037792.469 & 11927062037792.469 \\
sse & 157457720227.919 & 32215998366.736 & 903559871079.215 & 124824257960.295 \\
mse & 157457713.228 & 161079984.834 & 4517799354.396 & 624121285.801 \\
rmse & 12548.216 & 12691.729 & 67214.577 & 24982.420 \\
mae & 10078.898 & 10289.124 & 56408.842 & 20247.645 \\
m & 1000.000 & 200.000 & 200.000 & 200.000 \\
dfr & 7.000 & 7.000 & 1.000 & 4.000 \\
df & 992.000 & 192.000 & 198.000 & 195.000 \\
fStat & 58133.527 & 10549.192 & 2440.016 & 4727.542 \\
\bottomrule
\end{tabular}

\end{table}

%%%%%%%%%%%%%%%%%%%%%%% Medical Cost %%%%%%%%%%%%%%%%%%%%%%%
\subsection*{Transformed Regression with Sqrt, Log1p, and Box--Cox Transformations on the Medical Cost Dataset: In-Sample, Validation, Forward, and Backward Results}

Finally, for the medical cost dataset, the sqrt transformation explained the insurance costs better than the other methods, despite the relatively lower $R^{2}$ values. For the in-sample case, the $R^{2}$ values were 0.753, 0.523, and 0.576 for the sqrt, log1p, and Box-Cox transformations, respectively, with adjusted $R^{2}$ values of 0.751, 0.520, and 0.574. These results were verified by the validation set, where the sqrt transformation achieved an $R^{2}$ of 0.706 and an adjusted $R^{2}$ of 0.697, outperforming the log1p ($R^{2}$ of 0.505, Adjusted $R^{2}$ of 0.490) and Box-Cox (($R^{2}$ of 0.546, Adjusted $R^{2}$ of 0.532) models.

\begin{table}[h]
\centering
\caption{Medical Cost  Regression with Square-Root Transformation}
\label{tab:medical_cost_sqrt}
\begin{tabular}{lrrrr}
\toprule
 & In-Sample & Validation & Forward & Backward \\
\midrule
rSq & 0.753 & 0.706 & 0.702 & 0.704 \\
rSqBar & 0.751 & 0.697 & 0.699 & 0.699 \\
sst & 196074221568.367 & 35797001122.000 & 35797001122.000 & 35797001122.000 \\
sse & 48497528289.829 & 10509483431.295 & 10649651869.890 & 10582381967.672 \\
mse & 36246276.223 & 39214482.415 & 39737503.977 & 39486494.879 \\
rmse & 6020.489 & 6262.147 & 6303.769 & 6283.828 \\
mae & 3613.896 & 3611.073 & 3687.410 & 3606.738 \\
m & 1338.000 & 268.000 & 268.000 & 268.000 \\
dfr & 8.000 & 8.000 & 3.000 & 5.000 \\
df & 1329.000 & 259.000 & 264.000 & 262.000 \\
fStat & 508.937 & 80.606 & 210.946 & 127.713 \\
\bottomrule
\end{tabular}

\end{table}

\begin{table}[h]
\centering
\caption{Medical Cost  Regression with Log1p Transformation}
\label{tab:medical_cost_log1p}
\begin{tabular}{lrrrr}
\toprule
 & In-Sample & Validation & Forward & Backward \\
\midrule
rSq & 0.523 & 0.505 & -415.056 & -53.303 \\
rSqBar & 0.520 & 0.490 & -419.784 & -54.340 \\
sst & 196074221568.367 & 35797001122.000 & 35797001122.000 & 35797001122.000 \\
sse & 93570168847.798 & 17725546380.471 & 14893568669580.479 & 1943894999500.171 \\
mse & 69932853.620 & 66140090.435 & 55573017420.808 & 7253339545.374 \\
rmse & 8362.587 & 8132.656 & 235739.300 & 85166.540 \\
mae & 4219.512 & 4231.425 & 66672.183 & 25327.152 \\
m & 1338.000 & 268.000 & 268.000 & 268.000 \\
dfr & 8.000 & 8.000 & 3.000 & 5.000 \\
df & 1329.000 & 259.000 & 264.000 & 262.000 \\
fStat & 183.219 & 34.154 & -89.119 & -52.613 \\
\bottomrule
\end{tabular}

\end{table}

\begin{table}[h]
\centering
\caption{Medical Cost  Regression with Box-Cox Transformation}
\label{tab:medical_cost_box_cox}
\begin{tabular}{lrrrr}
\toprule
 & In-Sample & Validation & Forward & Backward \\
\midrule
rSq & 0.576 & 0.546 & -61.221 & -10.125 \\
rSqBar & 0.574 & 0.532 & -61.928 & -10.337 \\
sst & 196074221568.367 & 35797001122.000 & 35797001122.000 & 35797001122.000 \\
sse & 83083467025.754 & 16248889297.412 & 2227326392657.103 & 398225509782.019 \\
mse & 62095258.835 & 60630175.946 & 8310919372.586 & 1485916076.276 \\
rmse & 7880.055 & 7786.539 & 91164.244 & 38547.582 \\
mae & 4052.895 & 4083.531 & 31806.787 & 14527.915 \\
m & 1338.000 & 268.000 & 268.000 & 268.000 \\
dfr & 8.000 & 8.000 & 3.000 & 5.000 \\
df & 1329.000 & 259.000 & 264.000 & 262.000 \\
fStat & 227.454 & 40.302 & -87.898 & -48.782 \\
\bottomrule
\end{tabular}

\end{table}

\section{Comparing Different Models on the Same Data Set}

\subsection{Auto MPG}

\begin{table}[h]
\centering
\caption{Scalation - Auto MPG In-Sample QoF Comparison}
\label{tab:Scalation - AutoMPG In-Sample QoF Comparison}
\begin{tabular}{|c|c|c|c|c|c|} \hline 
Metric & Regression & Ridge & Lasso & Sqrt & log1p \\ \hline \hline 
rSq &0.809255 & 0.776580 &      0.809163 &      0.835138 &      0.849102 \\ \hline 
rSqBar &0.806283 &      0.772507 &      0.806189 &      0.832569 &      0.846751 \\ \hline 
sst &23819.0 &  23819.0 &       23819.0 &       23819.0 &       23819.0 \\ \hline 
sse &4543.35 &  5321.63 &       4545.54 &       3926.85 &       3594.23 \\ \hline 
sde &3.40878 &  3.68883 &       3.40888 &       3.16757 &       3.02514 \\ \hline 
mse0 &11.5902 & 13.5756 &       11.5958 &       10.0175 &       9.16895 \\ \hline 
rmse &3.40443 & 3.68451 &       3.40526 &       3.16504 &       3.02803 \\ \hline 
mae &2.61826 &  2.79509 &       2.61703 &       2.34675 &       2.18422 \\ \hline 
smape &12.0589 &        65.4181 &       11.9861 &       10.2046 &       9.32001 \\ \hline 
m &392.000 &    392.000 &       392.000 &       392.000 &       392.000 \\ \hline 
dfr &6.00000 &  7.00000 &       6.00000 &       6.00000 &       6.00000 \\ \hline 
df &385.000 &   384.000 &       385.000 &       385.000 &       385.000 \\ \hline 
fStat &272.234 &        190.677 &       272.072 &       325.048 &       361.067 \\ \hline 
aic &-1022.45 & -1051.45 &      -1022.55 &      -993.873 &      -976.527 \\ \hline 
bic &-994.656 & -1019.68 &      -994.750 &      -966.074 &      -948.728 \\ \hline 
\end{tabular}
\end{table}

\begin{table}[h]
\centering
\caption{Scalation - Auto MPG Out-of-Sample QoF Comparison}
\label{tab:Scalation - AutoMPG Out-of-Sample QoF Comparison}
\begin{tabular}{|c|c|c|c|c|c|} \hline 
Metric & Regression & Ridge & Lasso & Sqrt & log1p \\ \hline \hline 
rSq &0.822842 & 0.797491 &      0.822903 &      0.846480 &      0.852864 \\ \hline 
rSqBar &0.820081 &      0.793799 &      0.820143 &      0.844088 &      0.850571 \\ \hline 
sst &4731.23 &  4731.23 &       4731.23 &       4731.23 &       4731.23 \\ \hline 
sse &838.174 &  958.118 &       837.889 &       726.337 &       696.133 \\ \hline 
sde &3.29026 &  3.51709 &       3.28969 &       3.05724 &       2.97969 \\ \hline 
mse0 &10.7458 & 12.2836 &       10.7422 &       9.31202 &       8.92478 \\ \hline 
rmse &3.27808 & 3.50479 &       3.27752 &       3.05156 &       2.98744 \\ \hline 
mae &2.48735 &  2.62052 &       2.48643 &       2.11846 &       1.98665 \\ \hline 
smape &11.8858 &        61.9272 &       11.8808 &       9.00068 &       8.28831 \\ \hline 
m &78.0000 &    78.0000 &       78.0000 &       78.0000 &       78.0000 \\ \hline 
dfr &6.00000 &  7.00000 &       6.00000 &       6.00000 &       6.00000 \\ \hline 
df &385.000 &   384.000 &       385.000 &       385.000 &       385.000 \\ \hline 
fStat &298.034 &        216.030 &       298.158 &       353.804 &       371.939 \\ \hline 
aic &-189.284 & -192.500 &      -189.271 &      -183.700 &      -182.048 \\ \hline 
bic &-172.787 & -173.646 &      -172.774 &      -167.203 &      -165.551 \\ \hline 
\end{tabular}
\end{table}

\begin{table}[h]
\centering
\caption{Statsmodels - Auto MPG In-Sample QoF Comparison}
\label{tab:Statsmodels - Auto MPG In-Sample QoF Comparison}
\begin{tabular}{|c|c|c|c|c|c|}\hline
Metric & Regression & Ridge & Lasso & Sqrt & Log1p \\ \hline \hline
rSq & 0.8093 & 0.8037 & 0.6416 & 0.8477 & 0.8725 \\ \hline
rSqBar & 0.8063 & 0.8012 & 0.6369 & 0.8453 & 0.8705 \\ \hline
sst & 23818.9935 & 23818.9935 & 23818.9935 & 252.1610 & 41.1728 \\ \hline
sse & 4543.3470 & 4675.2421 & 8537.0473 & 38.4071 & 5.2483 \\ \hline
sde & 3.4352 & 3.4802 & 4.7028 & 0.3158 & 0.1168 \\ \hline
mse0 & 11.8009 & 11.9266 & 21.7782 & 0.0998 & 0.0136 \\ \hline
rmse & 3.4352 & 3.4535 & 4.6667 & 0.3158 & 0.1168 \\ \hline
mae & 2.6183 & 2.6269 & 3.6246 & 2.3467 & 2.1842 \\ \hline
smape & 12.0589 & 12.0433 & 16.2905 & 10.2046 & 9.3200 \\ \hline
m & 392.0000 & 392.0000 & 392.0000 & 392.0000 & 392.0000 \\ \hline
dfr & 6.0000 & 6.0000 & 6.0000 & 6.0000 & 6.0000\\ \hline
df & 385.0000 & 386.0000 & 386.0000 & 385.0000 & 385.0000\\ \hline
fStat & 272.2341 & 263.4262 & 115.1614 & 357.1178 &439.2179\\ \hline
aic &2086.9095 &983.6796 & 1219.7162 & 215.8245 & -564.3874\\ \hline
bic & 2114.7083 & 1007.5071 & 1243.5438 & 243.6234 & -536.5886\\ \hline
\end{tabular}
\end{table}

\begin{table}[h]
\centering
\caption{Statsmodels - Auto MPG Out-of-Sample QoF Comparison}
\label{tab:Statsmodels - Auto MPG Out-of-Sample QoF Comparison}
\begin{tabular}{|c|c|c|c|c|c|}\hline
Metric & Regression & Ridge & Lasso & Sqrt & Log1p \\ \hline \hline
rSq & 0.8107 & 0.7854 & 0.5925 & 0.8482 & 0.8720 \\ \hline
rSqBar & 0.8070 & 0.7707 & 0.5646 & 0.8452 & 0.8695 \\ \hline
sst & 19750.2199 & 4032.2061 & 4032.2061 & 208.0819 & 33.8141 \\ \hline
sse & 3738.2664 & 865.2680 & 1643.1024 & 31.5951 & 4.3287 \\ \hline
sde & 3.4952 & 3.4428 & 4.7443 & 0.3213 & 0.1189 \\ \hline
mse0 & 12.2166 & 10.9528 & 20.7988 & 0.1033 & 0.0141 \\ \hline
rmse & 3.4952 & 3.3095 & 4.5606 & 0.3213 & 0.1189 \\ \hline
mae & 2.5039 & 2.5877 & 3.7495 & 2.1509 & 1.9533 \\ \hline
smape & 12.3880 & 12.5974 & 17.6830 & 9.8913 & 8.7248 \\ \hline
m & 79.0000 & 79.0000 & 79.0000 & 79.0000 & 79.0000 \\ \hline
dfr & 6.0000 & 6.0000 & 6.0000 & 6.0000 & 6.0000\\ \hline
df & 306.0000 & 73.0000 & 73.0000 & 306.0000 & 306.0000\\ \hline
fStat & 218.4461 & 44.5308 & 17.6906 & 284.8805 &347.3956\\ \hline
aic &1678.5500 &201.0937 & 251.7566 & 184.4835 & -437.6799\\ \hline
bic & 1704.7734 & 215.3104 & 265.9733 & 210.7069 & -411.4565\\ \hline
\end{tabular}
\end{table}

\subsection{House Prices}

\begin{table}[h]
\centering
\caption{Scalation - Housing Prices In-Sample QoF Comparison}
\label{tab:Scalation - Housing In-Sample QoF Comparison}
\begin{tabular}{|c|c|c|c|c|c|} \hline 
Metric & Regression & Ridge & Lasso & Sqrt & log1p \\ \hline \hline 
rSq &0.998516 & 0.987398 &      0.998516 &      0.986152 &      0.922307 \\ \hline 
rSqBar &0.998507 &      0.987309 &      0.998507 &      0.986068 &      0.921838 \\ \hline 
sst &6.42325e+13 &      6.42325e+13 &   6.42325e+13 &   6.42325e+13 &   6.42325e+13 \\ \hline 
sse &9.53030e+10 &      8.09438e+11 &   9.53030e+10 &   8.89504e+11 &   4.99039e+12 \\ \hline 
sde &9767.21 &  28463.5 &       9767.21 &       29836.4 &       70604.5 \\ \hline 
mse0 &9.53030e+07 &     8.09438e+08 &   9.53030e+07 &   8.89504e+08 &   4.99039e+09 \\ \hline 
rmse &9762.32 & 28450.6 &       9762.32 &       29824.5 &       70642.7 \\ \hline 
mae &7747.66 &  24147.7 &       7747.66 &       24309.3 &       53070.5 \\ \hline 
smape &1.57791 &        23.5576 &       1.57791 &       4.86164 &       9.22319 \\ \hline 
m &1000.00 &    1000.00 &       1000.00 &       1000.00 &       1000.00 \\ \hline 
dfr &6.00000 &  7.00000 &       6.00000 &       6.00000 &       6.00000 \\ \hline 
df &993.000 &   992.000 &       993.000 &       993.000 &       993.000 \\ \hline 
fStat &111378 & 11103.9 &       111378 &        11785.5 &       1964.69 \\ \hline 
aic &-10591.2 & -11658.9 &      -10591.2 &      -11708.0 &      -12570.3 \\ \hline 
bic &-10556.9 & -11619.6 &      -10556.9 &      -11673.7 &      -12536.0 \\ \hline 
\end{tabular}
\end{table}

\begin{table}[h]
\centering
\caption{Scalation - Housing Prices Out-of-Sample QoF Comparison}
\label{tab:Scalation - Housing Out-of-Sample QoF Comparison}
\begin{tabular}{|c|c|c|c|c|c|} \hline 
Metric & Regression & Ridge & Lasso & Sqrt & log1p \\ \hline \hline 
rSq &0.998649 & 0.989193 &      0.998649 &      0.984018 &      0.906860 \\ \hline 
rSqBar &0.998641 &      0.989117 &      0.998641 &      0.983921 &      0.906297 \\ \hline 
sst &1.33700e+13 &      1.33700e+13 &   1.33700e+13 &   1.33700e+13 &   1.33700e+13 \\ \hline 
sse &1.80638e+10 &      1.44485e+11 &   1.80638e+10 &   2.13678e+11 &   1.24528e+12 \\ \hline 
sde &9501.56 &  26880.4 &       9501.56 &       32757.9 &       78673.0 \\ \hline 
mse0 &9.03190e+07 &     7.22424e+08 &   9.03189e+07 &   1.06839e+09 &   6.22641e+09 \\ \hline 
rmse &9503.63 & 26877.9 &       9503.63 &       32686.3 &       78907.6 \\ \hline 
mae &7484.48 &  22419.4 &       7484.48 &       26186.8 &       58148.8 \\ \hline 
smape &1.60847 &        19.6544 &       1.60847 &       5.19788 &       9.91629 \\ \hline 
m &200.000 &    200.000 &       200.000 &       200.000 &       200.000 \\ \hline 
dfr &6.00000 &  7.00000 &       6.00000 &       6.00000 &       6.00000 \\ \hline 
df &993.000 &   992.000 &       993.000 &       993.000 &       993.000 \\ \hline 
fStat &122330 & 12971.9 &       122330 &        10189.9 &       1611.39 \\ \hline 
aic &-2101.67 & -2307.60 &      -2101.67 &      -2348.73 &      -2525.00 \\ \hline 
bic &-2078.59 & -2281.21 &      -2078.59 &      -2325.64 &      -2501.91 \\ \hline 
\end{tabular}
\end{table}

\begin{table}[h]
\centering
\caption{Statsmodels - House Price In-Sample QoF Comparison}
\label{tab:Statsmodels - House Price In-Sample QoF Comparison}
\begin{tabular}{|c|c|c|c|c|c|}\hline
Metric & Regression & Ridge & Lasso & Sqrt & Log1p \\ \hline \hline
rSq & 0.9985 & 0.9906 & 0.9875 & 0.9855 & 0.9415 \\ \hline
rSqBar & 0.9985 & 0.9905 & 0.9875 & 0.9854 & 0.9411 \\ \hline
sst & 64232463468052.5469 & 64232463468052.5469 & 64232463468052.5469 & 29445342.0661 & 241.7492 \\ \hline
sse & 95249090298.3967 & 604344645977.1542 & 800169699033.4265 & 425598.9981 & 14.1461 \\ \hline
sde & 9798.8381 & 24669.9185 & 28386.7992 & 20.7131 & 0.1194 \\ \hline
mse0 & 96017228.1234 & 604344645.9772 & 800169699.0334 & 429.0312 & 0.0143 \\ \hline
rmse & 9798.8381 & 24583.4222 & 28287.2710 & 20.7131 & 0.1194 \\ \hline
mae & 7740.4301 & 20205.2499 & 24002.5173 & 24296.4685 & 52989.2927 \\ \hline
smape & 1.5774 & 4.1035 & 4.9163 & 4.8579 & 9.2122 \\ \hline
m & 1000.0000 & 1000.0000 & 1000.0000 & 1000.0000 & 1000.0000 \\ \hline
dfr & 7.0000 & 7.0000 & 7.0000 & 7.0000 & 7.0000\\ \hline
df & 992.0000 & 993.0000 & 993.0000 & 992.0000 & 992.0000\\ \hline
fStat & 95425.1583 & 14935.3572 & 11245.5195 & 9662.8803 &2280.1137\\ \hline
aic &21225.8831 &20233.6552 & 20514.3344 & 8907.3747 & -1404.4424\\ \hline
bic & 21265.1451 & 20268.0095 & 20548.6887 & 8946.6367 & -1365.1804\\ \hline
\end{tabular}
\end{table}

\begin{table}[h]
\centering
\caption{Statsmodels - House Price Out-of-Sample QoF Comparison}
\label{tab:Statsmodels - House Price Out-of-Sample QoF Comparison}
\begin{tabular}{|c|c|c|c|c|c|}\hline
Metric & Regression & Ridge & Lasso & Sqrt & Log1p \\ \hline \hline
rSq & 0.9985 & 0.9913 & 0.9884 & 0.9852 & 0.9402 \\ \hline
rSqBar & 0.9985 & 0.9910 & 0.9880 & 0.9851 & 0.9397 \\ \hline
sst & 51340367232368.6250 & 12891771417242.6445 & 12891771417242.6445 & 23544096.3448 & 193.6629 \\ \hline
sse & 75080425564.8099 & 112341107192.6465 & 149469076702.9745 & 347277.4580 & 11.5792 \\ \hline
sde & 9736.4530 & 24126.2984 & 27828.9629 & 20.9400 & 0.1209 \\ \hline
mse0 & 94798517.1273 & 561705535.9632 & 747345383.5149 & 438.4816 & 0.0146 \\ \hline
rmse & 9736.4530 & 23700.3278 & 27337.6185 & 20.9400 & 0.1209 \\ \hline
mae & 8174.5836 & 19629.7981 & 23012.0161 & 23553.0274 & 51521.6937 \\ \hline
smape & 1.6620 & 3.9691 & 4.6112 & 4.7594 & 9.0567 \\ \hline
m & 200.0000 & 200.0000 & 200.0000 & 200.0000 & 200.0000 \\ \hline
dfr & 7.0000 & 7.0000 & 7.0000 & 7.0000 & 7.0000\\ \hline
df & 792.0000 & 193.0000 & 193.0000 & 792.0000 & 792.0000\\ \hline
fStat & 77254.5038 & 3136.4045 & 2350.4760 & 7557.5143 &1779.1786\\ \hline
aic &16972.0728 &4043.2977 & 4100.4076 & 7144.9157 & -1102.0191\\ \hline
bic & 17009.5497 & 4066.3859 & 4123.4958 & 7182.3926 & -1064.5422\\ \hline
\end{tabular}
\end{table}

\subsection{Insurance Charges}

\begin{table}[h]
\centering
\caption{Scalation - Insurance Charges In-Sample QoF Comparison}
\label{tab:Scalation - Insurance In-Sample QoF Comparison}
\begin{tabular}{|c|c|c|c|c|c|} \hline 
Metric & Regression & Ridge & Lasso & Sqrt & log1p \\ \hline \hline 
rSq &0.750157 & 0.749962 &      0.750157 &      0.752656 &      0.527431 \\ \hline 
rSqBar &0.748842 &      0.748457 &      0.748842 &      0.751354 &      0.524943 \\ \hline 
sst &1.96074e+11 &      1.96074e+11 &   1.96074e+11 &   1.96074e+11 &   1.96074e+11 \\ \hline 
sse &4.89878e+10 &      4.90260e+10 &   4.89878e+10 &   4.84977e+10 &   9.26587e+10 \\ \hline 
sde &6053.11 &  6055.42 &       6053.11 &       6001.45 &       8316.89 \\ \hline 
mse0 &3.66127e+07 &     3.66413e+07 &   3.66127e+07 &   3.62464e+07 &   6.92516e+07 \\ \hline 
rmse &6050.84 & 6053.20 &       6050.84 &       6020.50 &       8321.76 \\ \hline 
mae &4179.54 &  4150.46 &       4179.54 &       3623.20 &       4215.04 \\ \hline 
smape &37.9722 &        67.8472 &       37.9722 &       27.9341 &       26.5034 \\ \hline 
m &1338.00 &    1338.00 &       1338.00 &       1338.00 &       1338.00 \\ \hline 
dfr &7.00000 &  8.00000 &       7.00000 &       7.00000 &       7.00000 \\ \hline 
df &1330.00 &   1329.00 &       1330.00 &       1330.00 &       1330.00 \\ \hline 
fStat &570.477 &        498.274 &       570.477 &       578.161 &       212.057 \\ \hline 
aic &-13533.8 & -13532.3 &      -13533.8 &      -13527.1 &      -13960.2 \\ \hline 
bic &-13492.2 & -13485.5 &      -13492.2 &      -13485.5 &      -13918.6 \\ \hline 
\end{tabular}
\end{table}

\begin{table}[h]
\centering
\caption{Scalation - Insurance Charges Out-of-Sample QoF Comparison}
\label{tab:Scalation - Insurance Charges Out-of-Sample QoF Comparison}
\begin{tabular}{|c|c|c|c|c|c|} \hline 
Metric & Regression & Ridge & Lasso & Sqrt & log1p \\ \hline \hline 
rSq &0.720005 & 0.719714 &      0.720005 &      0.730926 &      0.583107 \\ \hline 
rSqBar &0.718531 &      0.718027 &      0.718531 &      0.729510 &      0.580913 \\ \hline 
sst &4.06432e+10 &      4.06432e+10 &   4.06432e+10 &   4.06432e+10 &   4.06432e+10 \\ \hline 
sse &1.13799e+10 &      1.13917e+10 &   1.13799e+10 &   1.09360e+10 &   1.69438e+10 \\ \hline 
sde &6540.46 &  6543.82 &       6540.46 &       6396.98 &       7972.86 \\ \hline 
mse0 &4.26213e+07 &     4.26655e+07 &   4.26213e+07 &   4.09588e+07 &   6.34601e+07 \\ \hline 
rmse &6528.50 & 6531.89 &       6528.50 &       6399.91 &       7966.18 \\ \hline 
mae &4430.66 &  4429.38 &       4430.66 &       3868.65 &       4087.97 \\ \hline 
smape &40.0602 &        62.9178 &       40.0602 &       30.5895 &       27.5909 \\ \hline 
m &267.000 &    267.000 &       267.000 &       267.000 &       267.000 \\ \hline 
dfr &7.00000 &  8.00000 &       7.00000 &       7.00000 &       7.00000 \\ \hline 
df &1330.00 &   1329.00 &       1330.00 &       1330.00 &       1330.00 \\ \hline 
fStat &488.584 &        426.574 &       488.584 &       516.126 &       265.753 \\ \hline 
aic &-2708.17 & -2706.31 &      -2708.17 &      -2702.86 &      -2761.31 \\ \hline 
bic &-2679.47 & -2674.02 &      -2679.47 &      -2674.16 &      -2732.61 \\ \hline 
\end{tabular}
\end{table}

\begin{table}[h]
\centering
\caption{Statsmodels - Insurance Charges In-Sample QoF Comparison}
\label{tab:Statsmodels - Insurance Charges In-Sample QoF Comparison}
\begin{tabular}{|c|c|c|c|c|c|}\hline
Metric & Regression & Ridge & Lasso & Sqrt & Log1p \\ \hline \hline
rSq & 0.7509 & 0.2718 & 0.7469 & 0.7795 & 0.7680 \\ \hline
rSqBar & 0.7494 & 0.2680 & 0.7456 & 0.7782 & 0.7666 \\ \hline
sst & 196074221568.3671 & 196074221568.3671 & 196074221568.3671 & 3051091.5131 & 1130.1100 \\ \hline
sse & 48839532843.9219 & 142780871945.3538 & 49623455821.6671 & 672635.9707 & 262.2315 \\ \hline
sde & 6062.1023 & 10361.1794 & 6108.2624 & 22.4972 & 0.4442 \\ \hline
mse0 & 36749084.1564 & 106712161.3941 & 37087784.6201 & 506.1219 & 0.1973 \\ \hline
rmse & 6062.1023 & 10330.1579 & 6089.9741 & 22.4972 & 0.4442 \\ \hline
mae & 4170.8869 & 8134.7714 & 4115.6884 & 3613.8958 & 4219.5115 \\ \hline
smape & 37.8059 & 66.1576 & 34.1997 & 27.6903 & 26.2889 \\ \hline
m & 1338.0000 & 1338.0000 & 1338.0000 & 1338.0000 & 1338.0000 \\ \hline
dfr & 8.0000 & 8.0000 & 8.0000 & 8.0000 & 8.0000\\ \hline
df & 1329.0000 & 1330.0000 & 1330.0000 & 1329.0000 & 1329.0000\\ \hline
fStat & 500.8107 & 62.0533 & 490.6438 & 587.4216 &549.8054\\ \hline
aic &27113.5058 &24749.7939 & 23335.7320 & 12137.4774 & 1634.5362\\ \hline
bic & 27160.2962 & 24791.3854 & 23377.3235 & 12184.2678 & 1681.3266\\ \hline
\end{tabular}
\end{table}

\begin{table}[h]
\centering
\caption{Statsmodels - Insurance Charges Out-of-Sample QoF Comparison}
\label{tab:Statsmodels - Insurance Charges Out-of-Sample QoF Comparison}
\begin{tabular}{|c|c|c|c|c|c|}\hline
Metric & Regression & Ridge & Lasso & Sqrt & Log1p \\ \hline \hline
rSq & 0.7417 & 0.2951 & 0.7797 & 0.7701 & 0.7572 \\ \hline
rSqBar & 0.7398 & 0.2761 & 0.7738 & 0.7684 & 0.7554 \\ \hline
sst & 154436975468.4681 & 41606660039.7953 & 41606660039.7953 & 2399231.6774 & 888.0691 \\ \hline
sse & 39887119421.1600 & 29328394351.2134 & 9164643207.2381 & 551569.9183 & 215.5882 \\ \hline
sde & 6131.3858 & 10620.8058 & 5937.0555 & 22.8004 & 0.4508 \\ \hline
mse0 & 37593892.0086 & 109434307.2806 & 34196429.8778 & 519.8585 & 0.2032 \\ \hline
rmse & 6131.3858 & 10461.0854 & 5847.7714 & 22.8004 & 0.4508 \\ \hline
mae & 4181.1945 & 8328.6585 & 4069.5309 & 3556.9640 & 3888.4432 \\ \hline
smape & 40.0220 & 69.4557 & 35.5690 & 29.1245 & 25.7136 \\ \hline
m & 268.0000 & 268.0000 & 268.0000 & 268.0000 & 268.0000 \\ \hline
dfr & 8.0000 & 8.0000 & 8.0000 & 8.0000 & 8.0000\\ \hline
df & 1061.0000 & 260.0000 & 260.0000 & 1061.0000 & 1061.0000\\ \hline
fStat & 380.8792 & 13.6061 & 115.0471 & 444.2703 &413.6950\\ \hline
aic &21708.8072 &4976.9038 & 4665.1653 & 9736.7961 & 1340.3418\\ \hline
bic & 21753.5859 & 5005.6317 & 4693.8932 & 9781.5748 & 1385.1205\\ \hline
\end{tabular}
\end{table}














































\end{document}