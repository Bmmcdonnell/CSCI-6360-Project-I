\documentclass{article}
\usepackage{amsmath} % For advanced math features like \text, \frac, etc.
\usepackage[utf8]{inputenc} % To handle various characters
\begin{document}

\begin{table}[H]
\centering
\caption{Medical Cost (In-Sample): Sqrt, Log1p, and Box–Cox($\lambda$ = 0.04)}
\label{tab:Medical_Cost_(In-Sample)}
\begin{tabular}{|c|c|c|c|} \hline 
Metric & sqrt & log1p & box-cox($\lambda$=0.04) \\ \hline \hline 
rSq &0.752657 & 0.522782 &      0.576265 \\ \hline 
rSqBar &0.751168 &      0.519909 &      0.573715 \\ \hline 
sst &1.96074e+11 &      1.96074e+11 &   1.96074e+11 \\ \hline 
sse &4.84975e+10 &      9.35702e+10 &   8.30835e+10 \\ \hline 
sde &6001.71 &  8358.44 &       7870.39 \\ \hline 
mse0 &3.62463e+07 &     6.99329e+07 &   6.20953e+07 \\ \hline 
rmse &6020.49 & 8362.59 &       7880.06 \\ \hline 
mae &3613.90 &  4219.51 &       4052.90 \\ \hline 
smape &27.6903 &        26.2889 &       26.0851 \\ \hline 
m &1338.00 &    1338.00 &       1338.00 \\ \hline 
dfr &8.00000 &  8.00000 &       8.00000 \\ \hline 
df &1329.00 &   1329.00 &       1329.00 \\ \hline 
fStat &505.514 &        181.986 &       225.924 \\ \hline 
aic &-13525.1 & -13964.7 &      -13885.2 \\ \hline 
bic &-13478.3 & -13917.9 &      -13838.4 \\ \hline 
\end{tabular}
\end{table}

\end{document}