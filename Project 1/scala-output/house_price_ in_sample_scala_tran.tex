\documentclass{article}
\usepackage{amsmath} % For advanced math features like \text, \frac, etc.
\usepackage[utf8]{inputenc} % To handle various characters
\begin{document}

\begin{table}[H]
\centering
\caption{House Price (In-Sample): Sqrt, Log1p, and Box–Cox($\lambda$ = 0.85)}
\label{tab:House_Price_(In-Sample)}
\begin{tabular}{|c|c|c|c|} \hline 
Metric & sqrt & log1p & box-cox($\lambda$=0.85) \\ \hline \hline 
rSq &0.986164 & 0.922376 &      0.997549 \\ \hline 
rSqBar &0.986067 &      0.921828 &      0.997531 \\ \hline 
sst &6.42325e+13 &      6.42325e+13 &   6.42325e+13 \\ \hline 
sse &8.88711e+11 &      4.98598e+12 &   1.57458e+11 \\ \hline 
sde &29823.1 &  70573.3 &       12554.0 \\ \hline 
mse0 &8.88711e+08 &     4.98598e+09 &   1.57458e+08 \\ \hline 
rmse &29811.3 & 70611.5 &       12548.2 \\ \hline 
mae &24296.5 &  52989.3 &       10078.9 \\ \hline 
smape &4.85788 &        9.21218 &       2.12365 \\ \hline 
m &1000.00 &    1000.00 &       1000.00 \\ \hline 
dfr &7.00000 &  7.00000 &       7.00000 \\ \hline 
df &992.000 &   992.000 &       992.000 \\ \hline 
fStat &10100.8 &        1683.94 &       57668.5 \\ \hline 
aic &-11705.6 & -12567.9 &      -10840.3 \\ \hline 
bic &-11666.3 & -12528.6 &      -10801.0 \\ \hline 
\end{tabular}
\end{table}

\end{document}