\documentclass{article}
\usepackage{graphicx} % Required for inserting images
\usepackage[utf8]{inputenc}
\usepackage{fancyhdr}
\usepackage[shortlabels]{enumitem}
\usepackage{amsmath, amsfonts, amsthm, amssymb, mathrsfs}
\usepackage{tikz}
\usepackage{subcaption}
\usepackage{geometry}
\usepackage{stackengine}
\usepackage{tikz-cd}
\usetikzlibrary{cd}
\usetikzlibrary{calc}
\usepackage{lipsum}
\usepackage{adjustbox}
\usepackage{cleveref}
\numberwithin{equation}{subsection}

\usepackage{scalerel,stackengine}
\stackMath
\newcommand\reallywidehat[1]{%
\savestack{\tmpbox}{\stretchto{%
  \scaleto{%
    \scalerel*[\widthof{\ensuremath{#1}}]{\kern-.6pt\bigwedge\kern-.6pt}%
    {\rule[-\textheight/2]{1ex}{\textheight}}%WIDTH-LIMITED BIG WEDGE
  }{\textheight}% 
}{0.5ex}}%
\stackon[1pt]{#1}{\tmpbox}%
}

\newcommand{\crly}[1]{\left\{#1\right\}}
\newcommand{\sq}[1]{\left[#1\right]}
\newcommand{\cur}[1]{\left(#1\right)}
\newcommand{\lrangle}[1]{\left\langle#1\right\rangle}
\newcommand{\abs}[1]{\left|#1\right|}
\newcommand{\nrm}[1]{\left|\left|#1\right|\right|}
\newcommand{\ceil}[1]{\left\lceil#1\right\rceil}
\newcommand{\floor}[1]{\left\lfloor#1\right\rfloor}

\newcommand{\suml}[2]{\sum\limits_{#1}^{#2}}
\newcommand{\prodl}[2]{\prod\limits_{#1}^{#2}}
\newcommand{\bigcupl}[2]{\bigcup\limits_{#1}^{#2}}
\newcommand{\bigcapl}[2]{\bigcap\limits_{#1}^{#2}}
\newcommand{\liml}[1]{\lim\limits_{#1}}
\newcommand{\supl}[1]{\sup\limits_{#1}}
\newcommand{\infl}[1]{\inf\limits_{#1}}

\newcommand{\N}{\mathbb{N}}
\newcommand{\R}{\mathbb{R}}
\newcommand{\Z}{\mathbb{Z}}
\newcommand{\C}{\mathbb{C}}
\newcommand{\Q}{\mathbb{Q}}
\newcommand{\mf}[1]{\mathfrak{#1}}
\newcommand{\mc}[1]{\mathcal{#1}}
\newcommand{\del}{\partial}
%\newcommand{\ker}{\text{ker}}
\newcommand{\Ker}{\textup{Ker}}
\newcommand{\Coker}{\textup{Coker}}
\newcommand{\im}{\textup{im}}
\newcommand{\IM}{\textup{Im}}
\newcommand{\id}{\textup{id}}
\newcommand{\Hom}[2]{\textup{Hom}_{#1}\cur{#2}}
\newcommand{\ind}[2]{\textup{ind}_{#1}^{#2}}
\newcommand{\res}[2]{\textup{res}_{#1}^{#2}}
\newcommand{\Rind}[3]{\textup{R}^{#1}\textup{ind}_{#2}^{#3}}
\newcommand{\rRind}[4]{\textup{R}_{#1}^{#2}\textup{ind}_{#3}^{#4}}
\newcommand{\soc}{\textup{soc}}
\newcommand{\Ext}[2]{\textup{Ext}_{#1}^{#2}}
\newcommand{\Tor}[2]{\textup{Tor}_{#1}^{#2}}
\newcommand{\Tot}{\textup{Tot}}
\newcommand{\Mod}[1]{\textup{Mod}\cur{#1}}
%\newcommand{\mod}[1]{\textup{mod}\cur{#1}}
\newcommand{\St}{\textup{St}}
\newcommand{\Char}[1]{\textup{char}\cur{#1}}

%\newcommand{\a}{\alpha}
%\newcommand{\b}{\beta}
%\newcommand{\d}{\delta}
\newcommand{\e}{\varepsilon}
%\newcommand{\D}{\Delta}
%\newcommand{\E}{\varEpsilon}

\makeatletter
\newcommand*{\medcdot}{\mathpalette\medcdot@{0.5}}
\newcommand*\medcdot@[2]{\mathbin{\vcenter{\hbox{\scalebox{#2}{$\m@th#1\bullet$}}}}}
\makeatother

\makeatletter
\newcommand*{\bigcdot}{\mathpalette\bigcdot@{0.75}}
\newcommand*\bigcdot@[2]{\mathbin{\vcenter{\hbox{\scalebox{#2}{$\m@th#1\bullet$}}}}}
\makeatother

%\renewcommand{\sum}[2]{\sum\limits_{#1}^{#2}}

\renewcommand{\qedsymbol}{$\blacksquare$}

\newtheorem{theorem}{Theorem}[section]
\newtheorem{lemma}[theorem]{Lemma}
\newtheorem{proposition}[theorem]{Proposition}
\newtheorem{corollary}[theorem]{Corollary}
\newtheorem{conjecture}[theorem]{Conjecture}
\newtheorem{observation}[theorem]{Observation}
\newtheorem{claim}[theorem]{Claim}
\newtheorem{question}[theorem]{Question}
\newtheorem{exercise}[theorem]{Exercise}
\newtheorem{example}[theorem]{Example}

\newcommand{\sref}[1]{\textup{sequence}~(\ref{#1})}
\newcommand{\Sref}[1]{\textup{Sequence}~(\ref{#1})}

\newcommand{\crefdefpart}[2]{%
  \hyperref[#2]{\namecref{#1}~\labelcref*{#1}~\ref*{#2}}%
}

\newcommand{\Crefdefpart}[2]{%
  \nameCref{#1}~\hyperref[#2]{\labelcref*{#1}~\ref*{#2}}%
}

\usepackage[backend=bibtex, style=alphabetic, sorting=nyt, maxnames=99]{biblatex}


%%%%removes the autamitic 'p' when citing a page
\DeclareFieldFormat{postnote}{#1}
\DeclareFieldFormat{multipostnote}{#1}

\title{Project 1}
\author{Manager: Brendan McDonnel \phantom{thisisatest}\\
Masum Billah\\
Madhu Chencharapu\\
Gabriel Loos\\
Roshan Ravichandran}
\date{February 2026}

\begin{document}

\maketitle

\section{Auto MPG}

\begin{table}[h]
\centering
\caption{AutoMPG Linear Regression}
\label{tab:Linear Regression}
\begin{tabular}{|c|c|c|} \hline 
Regression & In-Sample & 80-20 Split \\ \hline \hline 
rSq &0.809255 & 0.808725 \\ \hline 
rSqBar &0.806283 &      0.805744 \\ \hline 
sst &23819.0 &  23819.0 \\ \hline 
sse &4543.35 &  4555.97 \\ \hline 
sde &3.40878 &  3.41318 \\ \hline 
mse0 &11.5902 & 11.6224 \\ \hline 
rmse &3.40443 & 3.40916 \\ \hline 
mae &2.61826 &  2.62805 \\ \hline 
smape &12.0589 &        12.2106 \\ \hline 
m &392.000 &    392.000 \\ \hline 
dfr &6.00000 &  6.00000 \\ \hline 
df &385.000 &   385.000 \\ \hline 
fStat &272.234 &        271.302 \\ \hline 
aic &-1022.45 & -1023.00 \\ \hline 
bic &-994.656 & -995.200 \\ \hline 
\end{tabular}
\end{table}

\begin{figure}[h]
    \centering
    \includegraphics[width=0.8\textwidth]{Auto_MPG/In_Sample.png}
    \caption{Auto MPG Linear Regression In Sample Predictions: yy black/actual vs. yp red/predicted}
    \label{fig:Auto MPG Linear Regression In Sample Predictions}
\end{figure}

\begin{figure}[h]
    \centering
    \includegraphics[width=0.8\textwidth]{Auto_MPG/80_20.png}
    \caption{Auto MPG Linear Regression 80-20 Validation Split Predictions: yy black/actual vs. yp red/predicted}
    \label{fig:Auto MPG Linear Regression 80-20 Validation Split Predictions}
\end{figure}

This text is here to force the images to be in the correct section.

\vspace{5cm}

This text is here to force the images to be in the correct section.

\vspace{5cm}

This text is here to force the images to be in the correct section.

\section{Housing Prices}

\begin{table}[h]
\centering
\caption{House Price Linear Regression}
\label{tab:Linear Regression}
\begin{tabular}{|c|c|c|} \hline 
Regression & In-Sample & 80-20 Split \\ \hline \hline 
rSq &0.998516 & 0.998515 \\ \hline 
rSqBar &0.998507 &      0.998506 \\ \hline 
sst &6.42325e+13 &      6.42325e+13 \\ \hline 
sse &9.53030e+10 &      9.54013e+10 \\ \hline 
sde &9767.21 &  9771.24 \\ \hline 
mse0 &9.53030e+07 &     9.54013e+07 \\ \hline 
rmse &9762.32 & 9767.36 \\ \hline 
mae &7747.66 &  7741.86 \\ \hline 
smape &1.57791 &        1.57811 \\ \hline 
m &1000.00 &    1000.00 \\ \hline 
dfr &6.00000 &  6.00000 \\ \hline 
df &993.000 &   993.000 \\ \hline 
fStat &111378 & 111264 \\ \hline 
aic &-10591.2 & -10591.7 \\ \hline 
bic &-10556.9 & -10557.4 \\ \hline 
\end{tabular}
\end{table}

\newpage

\begin{figure}[h]
    \centering
    \includegraphics[width=0.8\textwidth]{Housing/In_Sample.png}
    \caption{House Price Linear Regression In Sample Predictions: yy black/actual vs. yp red/predicted}
    \label{fig:House Price Linear Regression In Sample Predictions}
\end{figure}

\begin{figure}[h]
    \centering
    \includegraphics[width=0.8\textwidth]{Housing/80_20.png}
    \caption{House Price Linear Regression 80-20 Validation Split Predictions: yy black/actual vs. yp red/predicted}
    \label{fig:House Price Linear Regression 80-20 Validation Split Predictions}
\end{figure}

This text is here to force the images to be in the correct section.

\vspace{5cm}

This text is here to force the images to be in the correct section.

\vspace{5cm}

This text is here to force the images to be in the correct section.

\vspace{5cm}

This text is here to force the images to be in the correct section.

\section{Insurance Charges}

\begin{table}[h]
\centering
\caption{Insurance Charges Linear Regression}
\label{tab:Linear Regression}
\begin{tabular}{|c|c|c|} \hline 
Regression & In-Sample & 80-20 Split \\ \hline \hline 
rSq &0.750157 & 0.749834 \\ \hline 
rSqBar &0.748842 &      0.748518 \\ \hline 
sst &1.96074e+11 &      1.96074e+11 \\ \hline 
sse &4.89878e+10 &      4.90510e+10 \\ \hline 
sde &6053.11 &  6057.00 \\ \hline 
mse0 &3.66127e+07 &     3.66600e+07 \\ \hline 
rmse &6050.84 & 6054.75 \\ \hline 
mae &4179.54 &  4159.76 \\ \hline 
smape &37.9722 &        37.3810 \\ \hline 
m &1338.00 &    1338.00 \\ \hline 
dfr &7.00000 &  7.00000 \\ \hline 
df &1330.00 &   1330.00 \\ \hline 
fStat &570.477 &        569.497 \\ \hline 
aic &-13533.8 & -13534.6 \\ \hline 
bic &-13492.2 & -13493.1 \\ \hline 
\end{tabular}
\end{table}

\begin{figure}[h]
    \centering
    \includegraphics[width=0.8\textwidth]{Insurance/In_Sample.png}
    \caption{Insurance Charges Linear Regression In Sample Predictions: yy black/actual vs. yp red/predicted}
    \label{fig:Insurance Charges Linear Regression In Sample Predictions}
\end{figure}

\begin{figure}[h]
    \centering
    \includegraphics[width=0.8\textwidth]{Insurance/80_20.png}
    \caption{Insurance Charges Linear Regression 80-20 Validation Split Predictions: yy black/actual vs. yp red/predicted}
    \label{fig:Insurance Charges Linear Regression 80-20 Validation Split Predictions}
\end{figure}

\end{document}